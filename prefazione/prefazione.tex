\chapter*{Introduzione} 
\addstarredchapter{Introduzione} %NECESSARIO PER I MINITOC SE NO SI SBALLANO
Il presente lavoro di tesi è stato svolto presso la U.O.C. di Fisica Sanitaria della ASL di Teramo ed è consistito nella messa in uso clinico di un sistema di elaborazione di piani di trattamento per il calcolo della dose al paziente erogata tramite un acceleratore lineare recentemente acquisito dalla U.O.C di Radioterapia. Un acceleratore lineare è un'apparecchiatura elettromedicale capace di produrre un fascio di radiazione ionizzante (fotonico o elettronico) che viene impiegato per il trattamento di neoplasie di vario genere....


continua descrivendo il fatto che per il calcolo della dose serve un TPS