\chapter*{Introduzione} 
\addstarredchapter{Introduzione} %NECESSARIO PER I MINITOC SE NO SI SBALLANO
Il presente lavoro di tesi è stato svolto presso la Unità Operativa Complessa (U.O.C.) di Fisica Sanitaria della ASL di Teramo ed è consistito nella messa in uso clinico di un sistema di elaborazione di piani di trattamento per il calcolo della dose al paziente erogata tramite un acceleratore lineare recentemente acquisito dalla U.O.C di Radioterapia. 

Un acceleratore lineare (LINAC) è un'apparecchiatura elettromedicale capace di produrre un fascio di radiazione ionizzante (fotonico o elettronico) che viene impiegato per il trattamento di neoplasie di vario genere. In radioterapia la quantità di energia assorbita da un determinato tessuto a seguito del passaggio della radiazione ionizzante è denominata \textit{dose assorbita} e viene utilizzata al pari della dose farmacologica per realizzare un determinato effetto terapeutico sulla neoplasia.\\
Il processo che conduce all'erogazione del trattamento sul paziente è articolato in vari step gestiti da un team multidisciplinare che comprende personale medico, fisico, tecnico ed infermieristico. Nella fattispecie, il fisico medico è inquadrato nel contesto italiano dal D.lgs n.187/00 che stabilisce, tra le altre cose, la diretta responsabilità di questa figura per quanto riguarda gli aspetti dosimetrici del trattamento radioterapico al fine di erogare al paziente una definita dose ad un determinato target individuati dal medico specialista in radioterapia. Questo processo è noto più comunemente con il termine di \textit{pianificazione del trattamento radioterapico}. Gli strumenti e le metodologie per effettuare la pianificazione del trattamento hanno subito profonde evoluzioni e perfezionamenti grazie soprattutto alla rivoluzione tecnologica ed alla diffusione di sistemi di calcolo tramite computer. 

In questo contesto si colloca lo strumento oggetto di questa tesi denominato \textit{sistema di elaborazione di piani di trattamento} o TPS (dall'acronimo inglese di \textit{treatment planning system}). Questo sistema consiste in un pacchetto software che viene utilizzato dal fisico medico per effettuare la pianificazione del trattamento radioterapico sulla base delle indicazioni cliniche ricevute dal medico radioterapista. Nella fattispecie, tramite l'utilizzo di un TPS è possibile simulare varie condizioni di irradiazione e calcolare la dose al paziente, in modo da poter determinare la metodologia di trattamento più adatta al particolare caso clinico. Il medico radioterapista discute con il fisico medico le varie soluzioni di irradiazione pianificate e sceglie quella più adatta dal punto di vista del suo giudizio clinico e del fine terapeutico.

Tuttavia, prima di poter utilizzare clinicamente un TPS è necessario che il fisico medico ponga in essere tutta una serie di azioni che sono riassunte nella dicitura \textit{commissioning}. Queste azioni consistono nell'effettuare una serie di misurazioni di varia natura sul fascio di radiazione prodotto dal LINAC. Queste misurazioni vengono utilizzate per costruire un modello fisico-matematico del LINAC all'interno del TPS. Il modello così costruito viene poi verificato in varie situazioni che simulano gli scenari clinici per cui si vuole impiegare il TPS. Questo processo si effettua pianificando ed erogando dei trattamenti su fantocci che simulano dei pazienti e delle condizioni di irradiazione reali e misurando l'accordo tra la dose predetta dal TPS e quella effettivamente misurata nel fantoccio.

Lo scopo di questo lavoro di tesi è stato dunque quello di effettuare il commissioning del TPS RayStation, (RaySearch Labs, Svezia) per la pianificazione di trattamenti radioterapici effettuati con un LINAC modello Trilogy (Varian Medical Systems, California) in dotazione alla U.O.C. di Radioterapia della ASL di Teramo. Il sistema è stato posto in uso clinico per diverse tecniche di pianificazione ed erogazione della dose. Tali tecniche sono denominate (in ordine di complessità) \textit{radioterapia conformazionale o 3D-CRT}, \textit{radioterapia ad intensità modulata o IMRT} e \textit{radioterapia volumetrica ad arco o VMAT}.

Assieme alla messa in uso clinico delle citate tecniche di irradiazione, la fase finale di questo lavoro di tesi è consistita nell'estendere il commissioning del TPS ad una ulteriore funzionalità che offre RayStation nel campo della radioterapia adattiva. Nella radioterapia classica la personalizzazione del trattamento avviene solo a livello della pianificazione iniziale e poi si assume che le condizioni stabilite in questa fase non cambino per tutto il corso del trattamento. La radioterapia adattiva d'altra canto mette in campo ulteriori metodologie di calcolo per poter monitorare l'andamento del trattamento in ogni sua singola seduta ed eventualmente ri-personalizzare il piano di cura in itinere a seguito di cambiamenti delle condizioni iniziali (es. paziente dimagrito, tumore regresso etc.).\\
Il processo di implementazione della radioterapia adattiva con valutazione giornaliera della dose al paziente presenta tuttavia vari problemi che sono ancora argomento di dibattito nella comunità scientifica. Per questo motivo, per il commissioning del TPS per radioterapia adattiva si è scelto di aderire a due studi multi-centrici italiani in modo da studiare e mettere in atto soluzioni condivise che assicurino una corretta e sicura introduzione della pratica nella routine clinica. Al tempo di redazione di questo lavoro, gli studi multi-centrici risultano essere ancora in via di svolgimento per cui saranno presentati in questa tesi soltanto i risultati preliminari cui si è giunti sia a livello di indagine locale che di indagine multi-centrica.





