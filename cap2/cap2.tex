\chapter{Modellizzazione f{}isica del LINAC nel TPS RayStation}
\minitoc
\textsf{In questo capitolo verranno introdotti i concetti di dosimetria di base di un acceleratore lineare. A partire da queste misure, viene costruito un modello dosimetrico del LINAC all'interno del TPS. Viene poi discussa l'accuratezza richiesta per un calcolo dosimetrico al variare della tecnica di erogazione assieme alle scelte e i metodi adottati per raggiungerla.}



\section{Dosimetria di base di un LINAC}
Le misure necessarie a costruire un modello dosimetrico di un LINAC consistono in curve di dosimetria relativa e misure puntuali di dose (assoluta e relativa).\\
Storicamente queste misurazioni vengono effettuate con detector a camera a ionizzazione\footnote{La camera a ionizzazione è un rivelatore di radiazione ionizzante a gas. \`E costituita da due elettrodi che racchiudono un certo volume di aria. Al passaggio della radiazione, l'aria viene ionizzata e libera coppie di ioni. Grazie ad un campo elettrico applicato tra i due elettrodi gli ioni migrano fino a giungere sugli elettrodi provocando un segnalo in corrente rivelabile}. Tuttavia, con l'avvento delle tecniche di radioterapia avanzata (es. intensità modulata o stereotassi) sono stati introdotti una molteplicità di detector di nuova generazione per soddisfare le esigenze di accuratezza di misura per piccoli campi di irradiazione\footnote{Un campo quadrato è ritenuto \textit{piccolo} se inferiore a $4$x$4$ cm$^2$ \cite{Das2008}.}. Nella sezione seguente verranno presentati i concetti classici di dosimetria di un LINAC propedeutici alla modellizzazione di un TPS. Ci si soffermerà dapprima alla tecnica di irradiazione denominata \textit{radioterapia conformazionale}\footnote{La radioterapia conformazionale o 3D-CRT è una tecnica di irradiazione che  fa impiego di fasci di irradiazione collimati sul target per i quali la pianificazione ed il calcolo della dose vengono effettuati su uno studio di tomografia computerizzata del paziente.}. Le problematiche relative alla dosimetria a piccoli campi e alle tecniche di irradiazione avanzata verranno discusse a seguire. 

\subsection{Dosimetria relativa}
Per dosimetria relativa si intende tutta una serie di misure della dose non in termini assoluti (in Gray) ma in percentuale rispetto a uno o più riferimenti. In particolare le misure di dosimetria relativa necessarie alla modellizzazione di un TPS generico possono essere suddivise in tre catagorie:
\begin{itemize}
\item Curve di dose-profondità (PDD).
\item Profili di dose.
\item Output factor.
\end{itemize}
Queste misure vengono effettuate in un fantoccio cubico riempito di acqua (materiale più simile ai tessuti umani) dotato di un carrello motorizzato su cui viene montata la camera a ionizzazione che effettua le misure di dose (vedi Fig.\ref{fig:wphant}).
\begin{figure}
\centering
\includegraphics[width=.47\textwidth]{./cap2/wphant.jpg}
\includegraphics[width=.47\textwidth]{./cap2/wphant_pos.jpg}
\caption{Fantoccio per le misure di dosimetria assoluta e relativa (foto e suo posizionamento rispetto al LINAC).}
\label{fig:wphant}
\end{figure}

Le curve di dose-profondità si ottengono muovendo il detector lungo l'asse centrale del fascio (in direzione verticale) con un certo step (tipicamente 1 mm).\\
Per ogni punto, la camera a ionizzazione registra una certa dose che viene normalizzata rispetto ad una lettura di riferimento (tipicamente la lettura corrispondente alla massima dose lungo la verticale).

I profili di dose si ottengono in maniera analoga muovendo il detector in piani perpendicolari all'asse del fascio. Tipicamente vengono acquisiti profili di dose lungo due assi perpendicolari denominati \textit{inline} e \textit{crossline}\footnote{Considerando un paziente supino con la testa verso il gantry del LINAC, l'asse \textit{inline} corrisponde alla direzione testa-piedi e l'asse \textit{cross-line} alla direzione sinistra-destra.} a varie profondità.\\
\begin{figure}
\centering
\includegraphics[width=.67\textwidth]{./cap2/pdd_prof.png}
\includegraphics[width=.3\textwidth]{./cap2/of.png}
\caption{Tipici andamenti delle misure di dosimetria relativa: curva dose-profondità (PDD) (sinistra); profilo di dose (centro); output factor (destra).}
\label{fig:pdd_prof}
\end{figure}
Nella Fig.\ref{fig:pdd_prof} è possibile visualizzare gli andamenti tipici di una curva dose-profondità e di un profilo di dose.

Nelle curve di dose-profondità è possibile distinguere tre zone:
\begin{itemize}
\item \textit{Zona di build-up:} la prima parte ascendente della PDD in cui avvengono le prime interazioni dei fotoni nel mezzo. Gli elettroni vengono messi in moto da queste interazioni primarie e depositano energia più avanti con i processi descritti nel capitolo 1. Se non vi fossero gli elettroni di contaminazione del fascio, la dose nei primi millimetri di materiale sarebbe quasi nulla\footnote{Questo effetto è considerato un beneficio in radioterapia ed è noto anche come \textit{skin-sparing-effect.}}.
\item \textit{Massimo della PDD:} all'aumentare della profondità aumenta la produzione di particelle secondarie fino ad arrivare ad un equilibrio noto come equilibrio di particelle cariche (CPE). In queste condizioni il numero di particelle cariche uscenti da un volume è uguale al numero di particelle entranti.
\item \textit{Parte discendente della PDD:} aumentando ulteriormente la profondità di tessuto attraversato, il fascio primario decresce esponenzialmente e conseguentemente anche la dose segue questo andamento
\end{itemize}

Una distinzione simile può essere fatta per i profili di dose:
\begin{itemize}
\item \textit{Zona in-field:} la parte centrale del profilo compresa in un range di dose che va dal 100\% all'80\% rispetto alla massima dose registrata.
\item \textit{Penombra:} ai bordi del campo la dose scende rapidamente a causa della schermatura dei collimatori ed è definita come quella zona compresa tra l'80\% e il 20\% della dose massima
\item \textit{Zona out-of-field:} al di fuori delle dimensioni geometriche del campo sono presenti delle code a bassa dose che si estendono lateralmente anche al di sotto della zona schermata dai collimatori. Questa dose residua è dovuta ai processi di scatter della radiazione primaria e dei secondari. Corrisponde alla zona al di sotto del 20\% della dose massima del profilo.
\end{itemize}

Come vedremo in seguito, l'accuratezza di un calcolo dosimetrico tramite TPS viene valutata per ognuna delle zone sopra citate, con diversi livelli di tolleranza. 

Assieme alle PDD e ai profili viene anche effettuata una misura di dose puntuale al variare delle dimensioni del campo di irradiazione (\textit{output factor}). In particolare viene registrata la lettura di dose al centro del fascio ad una certa profondità (tipicamente 10 cm) e per una dimensione di campo di riferimento (10$x$10 cm$^2$). Viene quindi ripetuta la lettura al variare delle dimensioni del campo di irradiazione normalizzando il tutto alla lettura di riferimento. L'andamento che si ottiene tipicamente per un fascio di fotoni è riportato nella Fig.\ref{fig:pdd_prof}. L'output factor incrementa all'aumentare delle dimensioni del campo e ciò è dovuto principalmente ai processi di scatter che assumono più rilevanza e contribuiscono a maggiorare la lettura di dose al centro.

\subsection{Dosimetria assoluta}
Una volta concluse le misure di dosimetria relativa, è necessario effettuare una misura di dose assoluta che permette di trasformare tutte le curve percentuali in curve di dose reale (in unità di Gray). La misurazione della dose assoluta è regolata da specifici protocolli redatti da organizzazioni internazionali come l'AAPM (America) \cite{Almond1999} e la International Atomic Energy Agency (IAEA) (Europa) \cite{Andreo2006}.\\
Per il commissioning dell'acceleratore oggetto di questa tesi è stato seguito il protocollo IAEA TRS-398 secondo cui la dose assoluta può essere ottenuta con la seguente formula:
\begin{equation}
D = M\,N_{d,w}\,k_{Q,Q_0}\,k_{TP}\,k_h\,k_{pol}\,k_{sat}
\end{equation}
dove:
\begin{description}
\item[$M:$] lettura della ionizzazione in Coulomb (effettuata con la camera a ionizzazione).
\item[$N_{d,w}:$] coefficiente di taratura della camera a ionizzazione che converte il valore in Coulomb nella dose in Gray in acqua per un fascio di riferimento (solitamente generato da una sorgente di Co-60).
\item[$k_{Q,Q_0}:$] fattore correttivo che tiene conto della diversa distribuzione energetica (qualità) di un fascio clinico generato da un LINAC e che dipende dalla specifica camera a ionizzazione.
\item[$k_{TP}:$] fattore correttivo che tiene conto della diversa temperatura e pressione durante la misura rispetto alle condizioni di riferimento.
\item[$k_{H}:$] fattore correttivo che tiene conto della diversa umidità durante la misura rispetto alle condizioni di riferimento.
\item[$k_{pol}:$] fattore che tiene conto della diversa lettura di carica che si ottiene con una camera a ionizzazione applicando un voltaggio positivo o negativo tra i due elettrodi.
\item[$k_{sat}:$] fattore che tiene conto degli effetti di ricombinazione delle cariche all'interno del volume di aria della camera a ionizzazione.
\end{description}

\subsection{Misure richieste per la modellizzazione del TPS RayStation}
Come già citato nelle sezioni precedenti, le misure necessarie a modellizzare un LINAC all'interno del TPS RayStation consistono in curve dose-profondità, profili di dose a varie profondità, output factor e una misura di dose assoluta per ogni qualità del fascio da modellizzare.\\
Nella Tab.\ref{tab:meas} sono indicate le dimensioni di campo per le quali è consigliato acquisire PDD, profili e gli output factor.
\begin{table}
\begin{tabular}{ccc}
Campi consigliati (cm$^2$) & Campi opzionali (cm$^2$) & Profondità profili (cm)\\
\hline\hline
2x2 & 1x1 & 1.5\\
3x3 & 4x4 & 3.0\\
5x5 & 6x6 & 5.0\\
10x10 & 7x7 & 10.0\\
15x15 & 8x8 & 15.0\\
20x20 & 9x9 & 20.0\\
30x30 & 12x12 & \\
40x40 & 25x25 & \\
\hline\hline
\end{tabular}
\caption{Misure consigliate e opzionali per le curve di PDD, profili e output factor.}
\label{tab:meas}
\end{table}
Queste misure costituiscono solo un'indicazione e possono essere estese o ridotte a seconda della tecnica di irradiazione e dell'accuratezza che si vuole raggiungere nel calcolo della dose.


\section{Accuratezza richiesta per la 3D-CRT}
L'accuratezza richiesta per il calcolo della dose in radioterapia conformazionale si basa su criteri stabiliti agli inizi degli anni duemila da Van Dyk et al. e Venselaar et al. \cite{Dyk1993,Venselaar2001}. Da questi due lavori pioneristici sono scaturite un certo numero di raccomandazioni da parte di organi internazionali (AAPM, ESTRO, IAEA) che hanno stabilito i limiti di accettabilità per la precisione di un calcolo di dose tramite TPS \cite{Fraass1998,Mijnheer2004,IAEA430}.\\
Il protocollo adottato per gli scopi di questa tesi è il \textit{booklet no.7} della \textit{European SocieTy for Radiotherapy \& Oncology} (ESTRO). 
\begin{figure}
\centering
\includegraphics[width=.65\textwidth]{./cap2/Accuracy_zones.png}\\\vspace{.3cm}
\includegraphics[width=\textwidth]{./cap2/Accuracy_pdd_prof.png}
\caption{In alto: suddivisione delle varie zone di interesse su cui valutare l'accuratezza del calcolo di dose con la tolleranza specifica $\delta_i$, $i\in[1,4]$. In basso: tolleranze visualizzate sulle varie zone in interesse per una PDD e un profilo di dose. I valori numerici delle tolleranze sono indicati nella Tab.\ref{tab:tol}}
\label{fig:accuracy_zones}
\end{figure}

\begin{table}
\centering
\includegraphics[width=\textwidth]{./cap2/accuracy_tol.png}
\caption{Livelli di tolleranza per le varie zone indicate nella Fig.\ref{fig:accuracy_zones} e per vari livelli di complessità del calcolo. Riprodotta da \cite{Mijnheer2004}.}
\label{tab:tol}
\end{table}

Nello specifico, nella radioterapia tradizionale 3D-CRT è sufficiente effettuare misure fino al campo di dimensioni 5$x$5 cm$^2$. Nelle tecniche di irradiazione speciali (es. intensità modulata) è necessario includere le misure dei campi \textquotedblleft piccoli\textquotedblright{} (i.e. $<4x4$ cm$^2$ \cite{Das2008}) con delle problematiche di misurazione che discuteremo in seguito.

Una estensiva review dei detector e delle tecniche di misura consigliate per piccoli campi è stata recentemente pubblicata dal task group 120 dell'associazione americana dei fisici medici (AAPM) \cite{Low2011}.




