\chapter*{Conclusioni}
\addcontentsline{toc}{chapter}{Conclusioni}

In questo lavoro di tesi si è collaborato attivamente con il personale della U.O.C. di Fisica Sanitaria della ASL di Teramo per porre in uso clinico un sistema di elaborazione di piani di trattamento di recente acquisizione. Il succitato sistema è parte integrante di un rinnovamento a largo spettro delle apparecchiature in dotazione ed uso presso la U.O.C. di Radioterapia che comprende, tra le altre cose, un acceleratore lineare di nuova generazione capace di erogare trattamenti di elevata complessità.

I vari step e gli obiettivi raggiunti nel corso della tesi sono riassunti sommariamente nell'elenco seguente:

\begin{itemize}
\item \`E stato studiato l'algoritmo \textit{collapsed-cone-convolution} per il calcolo della dose al paziente nella sua formulazione originale e nella sua implementazione all'interno del TPS RayStation. Sulla base di ciò è stato possibile comprendere i limiti e le precisioni raggiungibili con un calcolo dosimetrico basato su questo algoritmo.

\item Sono state studiate e messe in opera varie tecniche di misura della dose con particolare riguardo alle problematiche attuali della misurazione della dose in condizioni lontane da quelle standard (piccoli campi di irradiazione e utilizzo di detector di nuova generazione).

\item Sono state studiate ed applicate le metodologie di modellizzazione di un fascio clinico all'interno del TPS RayStation.

\item \`E stato sviluppato un tool di analisi per validare la modellizzazione secondo le direttive indicate in protocolli emessi da organi internazionali (AAPM, IAEA, ESTRO).

\item Sono state effettuate misure post-modellizzazione del tipo \textit{end-to-end test} su casi clinici simulati al termine delle quali il sistema RayStation è stato posto in uso clinico.

\item Si è esteso il commissioning del TPS ai tool che permettono di effettuare una valutazione giornaliera della dose erogata e conseguentemente di valutare la necessità di modificare l'erogazione della dose in modo che si adatti ai cambiamenti del paziente (radioterapia adattiva). A questo proposito si è preso parte a due studi multi-centrici italiani per studiare ed implementare in maniera sicura e condivisa i vari step necessari all'implementazione clinica di questa metodica.

\end{itemize}





 