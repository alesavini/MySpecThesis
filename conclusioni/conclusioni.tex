\chapter*{Conclusioni}
\addcontentsline{toc}{chapter}{Conclusioni}

In questo lavoro di tesi si è collaborato attivamente con il personale della U.O.C. di Fisica Sanitaria della ASL di Teramo per porre in uso clinico un sistema di elaborazione di piani di trattamento di recente acquisizione da parte della U.O.C. di Radioterapia. Il succitato sistema è parte integrante di un rinnovamento a largo spettro delle apparecchiature in dotazione alla U.O.C. di Radioterapia che comprende, tra le altre cose, un acceleratore lineare di nuova generazione capace di erogare trattamenti di complessità pari allo stato dell'arte attualmente disponibile.

I vari step e gli obiettivi raggiunti nel corso della tesi sono riassunti sommariamente nell'elenco seguente:

\begin{itemize}
\item \`E stato studiato l'algoritmo \textit{collapsed-cone-convolution} per il calcolo della dose al paziente nella sua formulazione originale e nella sua implementazione all'interno del TPS RayStation. Sulla base di ciò è stato possibile comprendere i limiti e le precisioni raggiungibili con un calcolo dosimetrico basato su questo algoritmo.

\item Sono state studiate e messe in opera varie tecniche di misura della dose con particolare riguardo alle problematiche attuali della misurazione della dose in condizioni lontane da quelle standard (piccoli campi di irradiazione e utilizzo di detector di nuova generazione).

\item Sono state studiate ed applicate le metodologie di modellizzazione di un fascio clinico all'interno del TPS RayStation.

\item \`E stato sviluppato un tool di analisi per validare la modellizzazione secondo le direttive indicate in protocolli emessi da organi internazionali (AAPM, IAEA, ESTRO).

\item Sono state effettuate misure post-modellizzazione del tipo \textit{end-to-end test} su casi clinici simulati al termine delle quali il sistema RayStation è stato posto in uso clinico.

\item Si è esteso il commissioning del TPS ai tool che permettono di effettuare una valutazione giornaliera della dose erogata e conseguentemente di valutare la necessità di modificare l'erogazione della dose in modo che si adatti ai cambiamenti del paziente (radioterapia adattiva). A questo proposito si è preso parte a due studi multi-centrici italiani per studiare ed implementare in maniera sicura e condivisa i vari step necessari all'implementazione clinica di questa metodica.

\end{itemize}

\chapter*{Ringraziamenti}
Questo lavoro di tesi rappresenta la conclusione di un percorso durato più di quattro anni e che ha rappresentato un passo fondamentale ed indelebile della mia vita. Tutto ciò è stato possibile grazie alle persone che mi hanno accompagnato, istruito ed assistito in questo cammino. Le parole che vi rivolgo non rendono onore all'importanza che avete rivestito per me.

\vspace*{2ex}

Desidero ringraziare il dr.Giovanni Orlandi, direttore della U.O.C. di Fisica Sanitaria della ASL di Teramo sul piano umano e professionale, per avermi accolto nella sua unità, per avermi subito fatto sentire parte del gruppo, per tutto il tempo dedicato a guidarmi, istruirmi ed educarmi ad esercitare questa professione, per essere stato relatore di questa tesi, per tutti i pomeriggi spesi nelle correzioni e nei consigli e per l'umanità che mi ha sempre dimostrato in ogni occasione.

\vspace*{2ex}

Desidero inoltre ringraziare i dirigenti fisici (presenti ed ex) della U.O.C. di Fisica Sanitaria, dr.Nico Adorante, dr.ssa Floriana Bartolucci, dr.Christian Fidanza e dr.ssa Federica Rosica per aver contribuito in maniera determinante alla mia formazione, per tutto il tempo speso a dispensare consigli, a correggere i miei errori ed a trasferire il loro sapere e la loro esperienza alla mia persona.

\vspace*{2ex}

Ringrazio la dr.ssa Silvia Colacicchi direttore bla bla sempre disponibile bla bla e con essa tutti i professori del corso di specializzazione in Fisica Medica dell'Università dell'Aquila, per avermi formato bla bla bla...

\vspace*{2ex}

Ringrazio poi il direttore della U.O.C. di Radioterapia dr. Carlo D'Ugo...per avermi dato accesso bla bla

\vspace*{2ex}

Ringrazio il personale medico...con particolare al dr.Saverio Malara e la dr.ssa Milena di Genesio

\vspace*{2ex}

Ringrazio il personale tecnico per avermi accolto come in famiglia...

\vspace*{2ex}

Ringrazio i miei compagni di specializzazione....

\vspace*{2ex}

Ringrazio infine la mia famiglia con le stesse parole utilizzate nella dedica di questa tesi. Grazie per aver permesso ai miei sogni di divenire realtà.






 