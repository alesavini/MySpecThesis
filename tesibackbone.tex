\documentclass[a4paper,12pt,italian,twoside]{book}
\usepackage[italian]{babel}
\usepackage[utf8]{inputenc}
\usepackage{amssymb}
\usepackage{amsmath}
\usepackage{graphicx}
\usepackage{enumerate}
\usepackage{fancyhdr}
\usepackage[labelfont=bf]{caption}
\usepackage{minitoc}
\usepackage[binding=1.5cm]{layaureo}
\usepackage{threeparttable}
\usepackage{booktabs}
\usepackage{longtable}
\usepackage{multirow}
\usepackage{array}
\usepackage{color}
\usepackage{comment}
\usepackage{ulem}\normalem
\usepackage{textcomp}
\usepackage[perpage,symbol*]{footmisc}
\usepackage{epstopdf}
\usepackage{csquotes}

\usepackage[style=ieee, backend=biber, citestyle=numeric-comp]{biblatex}
\addbibresource{./bibliografia/Th.bib}
\addbibresource{./bibliografia/Th-Adapt.bib}

\hyphenation{MATLAB}
\hyphenation{RayStation}
\hyphenation{IMRT}
\hyphenation{VMAT}
\newcommand{\virg}[1]{\textquotedblleft \emph{#1}\textquotedblright{}}
\newcommand{\virgbf}[1]{\textquotedblleft \textbf{\emph{#1}}\textquotedblright{}}
\newcommand{\de}{\,\text{d}}
\newcommand{\RS}{RayStation\textsuperscript{\textcopyright}}
\newcommand{\arrstr}[1]{\renewcommand{\arraystretch}{#1}}

\setcounter{tocdepth}{3}
\setcounter{secnumdepth}{3}  %Per avere le subsubsection numerate

%%%%%%AMBIENTE ABSTRACT%%%%%%%
%\newcommand{\abstractmodificato}[1]{
%
%\vspace*{3cm}
%\begin{center}
%
%\textbf{Abstract\\}
%\begin{minipage}[t]{0.7\columnwidth}
%
%\begin{small}{#1}\end{small}
%
%\end{minipage}
%
%\end{center}
%} 


%%%%%%INTESTAZIONE ELEGANTE%%%%%%
\pagestyle{fancy}
% with this we ensure that the chapter and section
% headings are in lowercase.
\renewcommand{\chaptermark}[1]{%
\markboth{#1}{}}
\renewcommand{\sectionmark}[1]{%
\markright{\thesection\ #1}}
\fancyhf{} % delete current header and footer
\fancyhead[LE,RO]{\bfseries\thepage}
\fancyhead[LO]{\bfseries\rightmark}
\fancyhead[RE]{\bfseries\leftmark}
\renewcommand{\headrulewidth}{0.5pt}
\renewcommand{\footrulewidth}{0pt}
\addtolength{\headheight}{2.5pt} % space for the rule
\fancypagestyle{plain}{%
\fancyhead{} % get rid of headers on plain pages
\renewcommand{\headrulewidth}{0pt} }
%%%%%%INTESTAZIONE ELEGANTE%%%%%%

%%%%%%INIZIO DOCUMENTO%%%%%%
\begin{document}
\dominitoc % Initialization minitocs
  
\frontmatter
%%%%%%FORNTESPIZIO%%%%%%
\begin{titlepage}
\pagestyle{empty}
\begingroup
% INTESTAZIONE
\vspace*{-7\topskip}

%%%%%%%%%%%%%%%%%%%%%%%%%%%%%%%%%%%%%%%%%%%%%%%%%%%%
%  FIGURA!!!!%%%%%%%%%%%%%%%%%%%%%%%%%%%%%%%%%%%%%%%
%%%%%%%%%%%%%%%%%%%%%%%%%%%%%%%%%%%%%%%%%%%%%%%%%%%%

\begin{figure}[!h]
%centrare
\begin{center}
%resize in cm
\includegraphics[scale=0.15]{frontespizio/logo_aq.jpg}
\includegraphics[scale=0.25]{frontespizio/logo_dept.jpg}
\end{center}
\end{figure}

\vspace{-1cm}

\centering \expandafter{\Large{UNIVERSIT\`{A} DEGLI STUDI DELL'AQUILA}\par}
\begin{center}
\Large{\rm\expandafter{Dipartimento di Medicina clinica, Sanità pubblica, Scienze della vita e dell'ambiente}}\\
\large\expandafter{\textbf{Scuola di Specializzazione in Fisica Medica}}
\end{center}

\vspace{0.5cm}
\begin{center}
        {\large{Tesi di Specializzazione}\par}
\end{center}

% TITOLO
\vspace{1.5cm}
\begin{center}
        {\huge\bf \baselineskip=0.95em plus 1pt \expandafter{
        Commissioning di un Sistema di elaborazione di piani di trattamento radioterapici\\\vspace{.5cm}
        \LARGE dalla 3D-CRT alla Radioterapia Adattiva}}
\end{center}





\vspace{3.5cm}

\begin{tabular}{c p{2.5 cm}c c}
\vspace{0.2cm}
Specializzando & & Relatore  \\
\vspace{0.2cm}
\large{\textbf{Alessandro Savini}} & & Dr.ssa \large{\textbf{Federica Rosica}}\\
 &&  \\
\dotfill && \dotfill \vspace{0.9cm}\\
\vspace{0.2cm}
         && Co-Relatore\\
				\vspace{0.2cm}
         && Prof. \large{\textbf{Sandro Santucci}}\\
&& \\
&& \dotfill

\end{tabular}
\vspace{2.5 cm}



%CHIUSURA
\begin{center}

\large{Anno Accademico 2014-2015}
\vspace{-2.5cm}
\end{center}

\clearpage
\endgroup

\end{titlepage}

\thispagestyle{empty}

%%%%%%DEDICA%%%%%%
\chapter*{}
\begin{flushright}
\emph{Alla mia Famiglia che ha permesso\\ che i miei sogni divenissero realtà.\\\vspace{.8cm} Ai miei Insegnanti che hanno contribuito\\ alla mia realizzazione umana e professionale.}
\end{flushright}
\newpage
\thispagestyle{empty}

%%%%%%PREFAZIONE%%%%%%
\chapter*{Introduzione} 
\addstarredchapter{Introduzione} %NECESSARIO PER I MINITOC SE NO SI SBALLANO
Il presente lavoro di tesi è stato svolto presso la U.O.C. di Fisica Sanitaria della ASL di Teramo ed è consistito nella messa in uso clinico di un sistema di elaborazione di piani di trattamento per il calcolo della dose al paziente erogata tramite un acceleratore lineare recentemente acquisito dalla U.O.C di Radioterapia. 

Un acceleratore lineare (LINAC) è un'apparecchiatura elettromedicale capace di produrre un fascio di radiazione ionizzante (fotonico o elettronico) che viene impiegato per il trattamento di neoplasie di vario genere. In radioterapia la quantità di energia assorbita da un determinato tessuto a seguito del passaggio della radiazione ionizzante è denominata \textit{dose assorbita} e viene utilizzata al pari della dose farmacologica per realizzare un determinato effetto terapeutico sulla neoplasia.\\
Il processo che conduce all'erogazione del trattamento sul paziente è articolato in vari step gestiti da un team multidisciplinare di professionalità che comprende personale infermieristico, tecnico, medico e fisico. Nella fattispecie, il fisico medico è inquadrato nel contesto italiano dal D.lgs n.187/00 che stabilisce, tra le altre cose, la diretta responsabilità di questa figura per quanto riguarda il calcolo e l'impostazione dei parametri del LINAC in modo da erogare al paziente una determinata dose ad un determinato target che vengono prescritti dal medico specialista in radioterapia. Questo processo è noto più comunemente con il termine di \textit{pianificazione del trattamento radioterapico}. Gli strumenti e le metodologie per effettuare la pianificazione del trattamento radioterapico hanno subito profonde evoluzioni e perfezionamenti grazie soprattutto alla rivoluzione tecnologica ed alla diffusione di sistemi di calcolo tramite computer. 

In questo contesto si colloca lo strumento oggetto di questa tesi denominato \textit{sistema di elaborazione di piani di trattamento} o TPS (dall'acronimo inglese di \textit{treatment planning system}). Questo sistema consiste in un pacchetto software che viene utilizzato dal fisico medico per effettuare la pianificazione del trattamento radioterapico sulla base delle prescrizioni ricevute dal medico radioterapista. Nella fattispecie, tramite l'utilizzo di un TPS è possibile simulare varie condizioni di irradiazione e calcolare la dose al paziente, in modo da poter determinare la metodologia di trattamento più adatta al particolare caso clinico. Il medico radioterapista discute con il fisico medico le varie soluzioni di irradiazione pianificate e sceglie quella più adatta dal punto di vista del suo giudizio clinico e del fine terapeutico.

Tuttavia, prima di poter utilizzare un TPS clinicamente è necessario che il fisico medico ponga in essere tutta una serie di azioni che sono riassunte nella dicitura \textit{commissioning}. Queste azioni consistono nell'effettuare una serie di misurazioni di varia natura sul fascio di radiazione prodotto dal LINAC (processo noto come dosimetria del LINAC). Queste misurazioni vengono utilizzate per costruire un modello fisico-matematico del LINAC all'interno del TPS (processo noto come modellizzazione del TPS). Il modello così costruito viene poi verificato in varie situazioni che simulano gli scenari clinici per cui si vuole impiegare il TPS. Questo processo si effettua pianificando ed erogando dei trattamenti su fantocci che simulano dei pazienti e delle condizioni di irradiazione reale e misurando l'accordo tra la dose predetta dal TPS e quella effettivamente misurata nel fantoccio (processo noto come verifica della modellizzazione).

Lo scopo di questo lavoro di tesi è stato dunque quello di effettuare il commissioning del TPS RayStation, (RaySearch Labs, Svezia) per la pianificazione di trattamenti radioterapici effettuati con un LINAC modello Trilogy (Varian Medical Systems, California). Il sistema è stato posto in uso clinico per diverse tecniche di pianificazione ed erogazione della dose denominate (in ordine di complessità crescente) \textit{radioterapia conformazionale o 3D-CRT}, \textit{radioterapia ad intensità modulata o IMRT} e \textit{radioterapia volumetrica ad arco o VMAT}.

Assieme alla messa in uso clinico di queste tecniche di irradiazione....continua dicendo che il commissioning è stato esteso alla RT adattiva anche, di recente interesse bla bla per aumentare sempre più la personalizzazione e la precisione del trattamento radioterapico....

continua descrivendo il fatto che per il calcolo della dose serve un TPS

\tableofcontents

%%%%%%CORPO DOCUMENTO%%%%%%
\mainmatter
\chapter{L'algoritmo \emph{collapsed cone} e la sua implementazione in RayStation}
\setcounter{minitocdepth}{1}
\minitoc
\setcounter{minitocdepth}{2}
\textsf{In questo capitolo verrà descritto l'algoritmo di calcolo dosimetrico \textit{collapsed-cone-convolution} e la sua implementazione all'interno del sistema di elaborazione di piani di trattamento o treatment planning system (TPS) \RS. Ci si soffermerà in particolare sugli aspetti riguardanti le approssimazioni intrinseche dell'algoritmo assieme alle approssimazioni adottate in fase di implementazione nel TPS. Ciò è propedeutico alla comprensione dei limiti e delle precisioni raggiungibili durante la modellizzazione di un fascio clinico per trattamenti radioterapici che verrà discusso nei capitoli successivi.}

\section{La dose assorbita in radioterapia}
\label{sec:intro}
In radioterapia la \textit{dose assorbita} è quella quantità che viene utilizzata al pari della dose farmacologica per ottenere un determinato effetto terapeutico. Più precisamente, la definizione formale è fornita nel report ICRU n.85 \cite{ICRU85} come rapporto tra l'energia media $\de \bar{\varepsilon}$ impartita da radiazioni ionizzanti ad una massa $\de m$:
\begin{equation}
D = \frac{\de \bar{\varepsilon}}{\de m} \qquad\qquad \text{Unità: J\,kg}^{-1} \equiv \text{Gray [Gy]}
\end{equation} 
Esistono varie modalità di impartire una certa dose ad un paziente in radioterapia. Nell'ambito di questo lavoro si considererà solo la tecnica che fa impiego di fotoni generati da un acceleratore lineare (LINAC) denominata \virg{radioterapia a fasci esterni}.

Un LINAC è un'apparecchiatura in grado di accelerare elettroni fino ad energie dell'ordine dei 20 MeV che vanno a collidere su un target da cui si origina radiazione di frenamento (bremsstrahlung). Il fascio di fotoni così generato viene opportunamente filtrato e collimato per generare un fascio terapeutico. 
\begin{figure}
\centering
\includegraphics[width=.7\textwidth]{./cap1/linac.png}
\caption{Figura schematica di un acceleratore lineare per radioterapia a fasci esterni.}
\label{fig:linac}
\end{figure}
Tutto ciò si realizza nella testata del LINAC mediante l'uso di opportuni materiali schermanti che sono indicati nel disegno schematico riportato in Fig.\ref{fig:linac}.

 \`{E} importante notare che una parte non trascurabile di processi che vanno ad influenzare il fascio che effettivamente giunge al paziente avviene a livello della testata. Ad esempio, uno degli effetti più clinicamente rilevanti è la generazione di elettroni che \textquotedblleft contaminano\textquotedblright{} il fascio fotonico.

Una volta che il fascio clinico investe il paziente, il meccanismo di deposizione della dose è un processo molto complesso dovuto alla grande quantità di fenomeni che vengono innescati in cascata.

\begin{figure}
\centering
\includegraphics[width=.9\textwidth]{./cap1/processes.png}
\caption{Rappresentazione schematica delle principali interazioni che portano alla deposizione della dose nel paziente.}
\label{fig:processes}
\end{figure}
\vspace{.2cm}
La Fig.\ref{fig:processes} riassume schematicamente le principali interazioni che portano alla deposizione della dose nel paziente. \`{E} possibile identificare quattro principali meccanismi di rilascio della dose (evidenziati nella figura) che vengono elencati di seguito:
\begin{enumerate}
\item La dose primaria che rappresenta generalmente fino al 70\% della dose totale. Questa dose è generata dalla parte di fascio fotonico che non ha subito trasformazioni nella testata e che mette in moto particelle cariche le quali direttamente rilasciano la loro energia cinetica nella materia.
\item La dose di scatter dovuta alla testata (\textit{head scatter dose}) che rappresenta generalmente il 5-10\% della dose totale. Questa parte della dose è dovuta alla componente di fascio fotonico che ha subito interazioni nella testata (prevalentemente nel \textit{flattening-filter}\footnote{\label{foot:flatt} Il \textit{flattening-filter} è un dispositivo di forma piramidale che serve ad attenuare il fascio al centro in modo da realizzare una fluenza di fotoni uniforme lungo la direzione perpendicolare alla direzione di propagazione del fascio.}) e presenta una distribuzione spaziale ed energetica differente dal fascio primario. Il meccanismo di rilascio dell'energia è analogo a quello del fascio primario.
\item La dose di scatter dovuta al paziente (\textit{phantom scatter dose}) che può rappresentare fino al 30\% della dose totale. Questa componente è dovuta a tutti i processi di scatter che si innescano nel paziente a partire dal fascio primario come ad esempio fotoni di bremsstrahlung o fotoni scatterati per effetto Compton che portano ad una ionizzazione della materia con lo stesso meccanismo del fascio primario (messa in moto di elettroni).
\item La dose dovuta alle particelle di contaminazione del fascio fotonico (elettroni). Queste particelle vengono prodotte nella testata del linac sia a partire dal target (assieme ai fotoni primari) sia dall'interazione dei fotoni primari stessi con il flattening-filter. Queste particelle hanno un effetto dosimetrico rilevante (comparabile con il fascio primario) soltanto nei primi centimetri di tessuto. Questa zona è conosciuta come \textit{regione di accrescimento della dose o di build-up} della dose.
\end{enumerate}


\section{Generalità sugli algoritmi di calcolo della dose al paziente per fasci di fotoni}
Un algoritmo di calcolo dosimetrico ha lo scopo di predire gli effetti di interazione radiazione-materia con un determinato livello di accuratezza. In particolare, il fine ultimo è predire la distribuzione di dose totale assorbita nel paziente che costituisce l'entità correlata all'effetto terapeutico sul tumore o al danno sul tessuto sano. La possibilità di prevedere questa quantità è propedeutica al processo noto come \textit{pianificazione del trattamento} in cui vengono adoperate delle opportune scelte riguardanti la collimazione e l'intensità del fascio volte a minimizzare il rapporto rischio/beneficio della terapia.

Esistono due grandi classi di algoritmi dosimetrici:
\begin{itemize}
\item Algoritmi \textit{correction-based}.
\item Algoritmi \textit{model-based}.
\end{itemize}
Gli algoritmi correction-based sono algoritmi empirici. Essi sono  basati su un gruppo di dati misurati in certe condizioni di riferimento e fanno uso di fattori o funzioni matematiche di tipo analitico o di tipo look-up-table per predire la distribuzione di dose assorbita in altre condizioni. 
Questi metodi furono i primi ad essere implementati in quanto non necessitano di grosse potenze di calcolo ma, d'altro canto, presentano dei limiti di accuratezza intrinseci per situazioni complesse (mezzi non omogenei, interfacce tra tessuti, campi di irradiazione molto irregolari o ad intensità modulata\ldots). Un'estensiva review di questi tipi di algoritmi è stata pubblicata da Fraass \textit{et al.}\cite{Fraass1995}.

\vspace{.2cm}
L'avvento della rivoluzione tecnologica e la crescita della potenza di calcolo disponibile, ha permesso l'implementazione degli algoritmi model-based i quali simulano i processi di interazione radiazione-materia  a partire da principi primi tramite un modello fisico-matematico.\\
In questo caso il set di misure iniziali è unicamente utilizzato per ottimizzare i parametri del modello che poi viene applicato per predire la distribuzione di dose assorbita nei vari scenari clinici. Questi algoritmi hanno dimostrato una maggiore accuratezza rispetto ai correction-based ed al giorno d'oggi i più utilizzati sono quelli basati su metodi semi-analitici (algoritmi di convolution/superposition) oppure algoritmi statistici basati su metodo Monte Carlo.

Il TPS RayStation in particolare implementa un algoritmo di tipo convolution/superposition conosciuto come \textit{collapsed cone convolution} sviluppato, a partire dalla metà dagli anni '80, indipendentemente da Mackie e Ahnesj{\"{o} \cite{Ahnesjo1989, Boyer1998, Mackie1985, Ahnesjo1987}.



\section{La sovrapposizione e la convoluzione in termini matematici}
Le operazioni di sovrapposizione e di convoluzione sono concetti matematici largamente utilizzati in fisica. Ragionando unicamente in termini matematici, l'operazione di sovrapposizione consiste nella combinazione lineare di una serie di funzioni $f_i(x)$, ognuna con un proprio peso $c_i$:
\begin{equation}
g(x) = \sum_{i=1}^{N} c_i\cdot f_i(x)
\end{equation}
Un tipico esempio fisico dell'operazione di sovrapposizione è il calcolo del campo elettrico in un punto dovuto ad una certa distribuzione di cariche puntiformi:
\begin{equation}
\mathbf E_0(\mathbf r) = \sum_{i=1}^N \mathbf E_{0i}(r) = \frac {1}{4 \pi \varepsilon_0} \sum_{i=1}^N q_i \frac {\mathbf r - \mathbf r_i'} {\left \| \mathbf r - \mathbf r_i' \right \|^3}
\end{equation}
Nel caso continuo, l'operazione di sovrapposizione può essere espressa con un integrale, esteso a tutto il dominio di definizione del problema, tra una funzione primaria $p$ ed una funzione coefficiente $s$ come funzione kernel:
\begin{equation}
\label{eq:sovrapp_math}
D(x,y,z) = \int_V p(x',y',z')\, s(x,x',y,y',z,z')\de x' \de y' \de z'
\end{equation}
Un particolare caso dell'operazione di sovrapposizione è rappresentato dalla convoluzione che si realizza quando la funzione kernel è spazialmente invariante, ovvero dipende solo dalla differenza tra la coordinata $(x,y,z)$ e la variabile di integrazione $(x',y',z')$:
\begin{equation}
\label{eq:convol_math}
D(x,y,z) = \int_V p(x',y',z')\, s(x-x',y-y',z-z')\de x' \de y' \de z'
\end{equation}


\section{Il calcolo della dose come sovrapposizione di eventi}
\label{sec:teoria_conv}
Il problema del calcolo della dose in un punto all'interno di un volume è in generale un problema di sovrapposizione di eventi che può essere tradotto matematicamente con l'Eq.\eqref{eq:sovrapp_math}.
\begin{figure}
\centering
\includegraphics[width=.45\textwidth]{./cap1/superp1.png}
\includegraphics[width=.45\textwidth]{./cap1/superp2.png}
\caption{Il calcolo della dose visto come sovrapposizione di eventi.}
\label{fig:superp}
\end{figure}

Nello specifico, osservando la Fig.\ref{fig:superp}, si nota come la quantità di energia che viene assorbita in un punto $P$ dipenda da infiniti contributi dovuti alle particelle ionizzanti messe in moto dai fotoni nei loro rispettivi centri di interazione $P'$.\\
La distribuzione dell'energia rilasciata dai fotoni nei centri di interazione primaria $P'$ va a formare una quantità denominata TERMA (total-energy-released-in-matter).  \\
Il TERMA è esprimibile come il prodotto tra la \textit{fluenza di energia primaria} (quantità di energia radiante incidente per unità di superficie [J m$^{-2}$]) e il coefficiente di assorbimento lineare massico del mezzo $(\mu/\rho)$ \cite{Ahnesjo1987}.

Introducendo la funzione di scatter\footnote{Funzione conosciuta anche come \textit{energy deposition point kernel} o \textit{point spread kernel} o \textit{kernel di deposizione}.} $s(x'\rightarrow x, y'\rightarrow y, z'\rightarrow z)$ che esprime l'ammontare di energia assorbita nel punto $P(x,y,z)$ dovuta all'energia rilasciata nell'interazione avvenuta nel punto $P'(x',y',z')$, la dose assorbita nel punto $P$ è esprimibile con un integrale:
\begin{align}
D(x,y,z) &=  \int_V \frac{\mu}{\rho} \Psi(x',y',z')\,s(x'\rightarrow x, y'\rightarrow y, z'\rightarrow z)\, \de x' \de y' \de z'\\
         &= \int_V T(x',y',z')\,s(x'\rightarrow x, y'\rightarrow y, z'\rightarrow z)\, \de x' \de y' \de z'\\
         &= \int_V T(P')\,s(P'\rightarrow P)\, \de V'
\end{align}

Questa equazione è valida nel caso di un fascio di fotoni monoenergetico in un mezzo omogeneo di densità $\rho$. La generalizzazione al caso polienergetico si effettua introducendo il TERMA differenziale in energia $T(P',E)$ e la funzione di scatter polienergetica $s(P'\rightarrow P,E)$ ed integrando su tutte le energie coinvolte:
\begin{equation}
D(P) = \iint_{E,V} T(P',E)\,s(P'\rightarrow P,E)\, \de V' \de E
\label{eq:superp}
\end{equation}
L'equazione \eqref{eq:superp} matematicamente è un'operazione di sovrapposizione, così come espresso nell'Eq.\eqref{eq:sovrapp_math} e rispecchia il fatto per cui la dose calcolata in un punto $P$ è la sovrapposizione di tanti fenomeni che avvengono nei punti $P'$. In aggiunta, considerando il caso di un mezzo omogeneo ed infinitamente esteso, la funzione di scatter risulta spazialmente invariante per cui l'\eqref{eq:superp} è un'equazione di convoluzione del tipo indicato nella \eqref{eq:convol_math}. Questa condizione matematica rispecchia il fenomeno fisico cosiddetto dell'\textit{equilibrio di particelle cariche}\footnote{L'equilibrio di particelle cariche (CPE) sussiste all'interno di un volume qualora il numero di particelle ionizzanti che entrano in esso, è uguale al numero di particelle che ne fuoriesce.}.

%Verrà dimostrato nelle sezioni successive come l'Eq.\eqref{eq:superp} possa essere scritta nella forma:
%\begin{equation}
%\boxed{D(\vec{r}) = \iint_{E,V} T_E(\rho_{\vec{r'}} \cdot \vec{r})\, s_E(\rho_{\vec{r}-\vec{r'}}\cdot (\vec{r}-\vec{r'}))\de\vec{r'} \de E}
%\label{eq:superp_true}
%\end{equation}
%che rappresenta la cosiddetta \textit{equazione di convolution/superposition} per il calcolo della dose in un volume disomogeneo investito da un fascio di fotoni polienergetico.



\section{Il calcolo della dose in RayStation}
\label{sec:algo_Ray}
Il calcolo della dose nel TPS RayStation fa uso del formalismo illustrato nelle sezioni precedenti e procede in quattro principali passaggi:
\begin{enumerate}
\item Il calcolo della fluenza di energia.
\item Il calcolo del TERMA.
\item L'applicazione delle opportune funzioni di scatter e del principio di \textit{convolution-superposition} per il calcolo della dose finale.
\item La somma del contributo dovuto alle particelle di contaminazione (elettroni).
\end{enumerate}

\subsection{Il calcolo della fluenza di energia}
\label{sec:fluence}
Questo primo step consiste in un calcolo geometrico che non tiene conto della presenza del paziente. Si è già notato nella sezione introduttiva (Fig.\ref{fig:processes}) come il fascio in ingresso in un paziente sia costituito da una parte primaria e da una parte che ha interagito con gli elementi della testata (in particolare con il flattening-filter). RayStation tratta queste due componenti con un modello a due sorgenti poste ad una certa distanza lungo la direzione di propagazione del fascio (Fig.\ref{fig:twosources}a).
\begin{figure}
\centering
a)\includegraphics[width=.5\textwidth]{./cap1/twosources.png}
b)\includegraphics[width=.4\textwidth]{./cap1/source_int.png}
\caption{(a) modello a due sorgenti utilizzato per il calcolo della fluenza di energia. (b) processo di integrazione della parte visibile della sorgente per il calcolo della fluenza.}
\label{fig:twosources}
\end{figure}
La sorgente primaria è modellizzata con un profilo gaussiano ellittico esteso dell'ordine dei mm mentre la sorgente di scatter del flattening-filter è gaussiana circolare estesa dell'ordine dei centimetri \cite{Chaney1994}.\\
Il calcolo della fluenza viene effettuato su un piano passante per il centro di simmetria rotazionale del LINAC denominato \textit{isocentro} e perpendicolare alla direzione del fascio (Fig.\ref{fig:twosources}a).
Le sorgenti vengono geometricamente proiettate attraverso i collimatori indicati in Fig.\ref{fig:linac} costituiti da blocchi di materiale schermante (\textit{jaws}) e da un dispositivo fatto di lamelle retraibili (\textit{multi-leaf-collimator}) che serve a generare conformazioni irregolari del fascio.\\
Matematicamente, l'operazione di calcolo della fluenza consiste in un'integrazione pixel per pixel della parte di sorgente \textquotedblleft visibile\textquotedblright{} attraverso i collimatori (Fig.\ref{fig:twosources}b) con un metodo di backprojection.\\
La mappa di fluenza così ottenuta viene corretta per includere alcuni fenomeni come la trasmissione dei collimatori, la trasmissione della punta e del bordo delle lamelle (\textit{leaf tip e tongue}\&\textit{groove}), il peso relativo delle sorgenti ed altri processi che verrano discussi in seguito.

In parallelo viene computata la mappa di fluenza per le sorgenti di elettroni di contaminazione (generati secondo le modalità descritte nella Sez.\ref{sec:intro}). Queste ultime sono gaussiane circolari e poste alla stessa posizione delle sorgenti di fotoni. Esse sono divise in primaria e di scatter del flattening filter e la loro intensità è espressa in percentuale rispetto alla fluenza dei fotoni primari.

\subsection{Il calcolo del TERMA}
Il TERMA costituisce la prima parte dell'integrale per il calcolo della dose assorbita (Eq.\ref{eq:superp}). Esso quantifica l'assorbimento della fluenza di energia che attraversa il paziente. A partire da questo assorbimento vengono applicate le funzioni di scatter per generare la distribuzione di dose.\\
Il TERMA differenziale in energia può essere scritto nella seguente forma generale:
\begin{equation}
\label{eq:termaE}
T(\vec{r'},E) = \frac{\mu}{\rho}(\vec{r'},E)\,\Psi(\vec{r'},E)
\end{equation}
dove $\vec{r'}$ è la coordinata del punto di interazione primaria, $\mu/\rho$ il coefficiente di assorbimento lineare massico, $\Psi$ la fluenza di energia primaria ed $E$ l'energia del fascio primario.

\begin{figure}
\centering
a)\includegraphics[width=.45\textwidth]{./cap1/TERMA_isoPlane.eps} b)
\includegraphics[width=.45\textwidth]{./cap1/TERMA_isoPatient.eps}
\caption{(a) calcolo della fluenza sul piano isocentrico senza considerare il paziente. (b) proiezione della fluenza sul piano parallelo al piano isocentrico e corrispondente alla superficie del paziente (ad una distanza sorgente-superficie del paziente (SSD).}
\label{fig:terma}
\end{figure}
Il paziente è modellizzato all'interno del TPS tramite uno studio di tomografia computerizzata che contiene una mappa di densità dei tessuti. Questo permette di conoscere il primo termine dell'Eq.\eqref{eq:termaE} \cite{RaySearchLaboratories2014}.\\
Il calcolo del secondo termine della \eqref{eq:termaE} avviene seguendo questi passaggi:
\begin{itemize}
\item Si parte dalla mappa di fluenza calcolata sul piano passante per l'isocentro e perpendicolare alla direzione di propagazione del fascio senza considerare la presenza del paziente (Fig.\ref{fig:terma}a).
\item La mappa di fluenza calcolata in assenza del paziente viene proiettata e riscalata verso la sorgente fino ad una distanza pari a $\vec{r_0}$ ove $\vec{r_0}$ costituisce la distanza dalla sorgente primaria alla superficie del paziente (distanza nota come Source-Surface-Distance o SSD) (Fig.\ref{fig:terma}b).
\item La mappa di fluenza calcolata all'ingresso del paziente $\Psi(\vec{r_0},E)$ viene rimodulata all'interno di questi (studio CT) tenendo conto di due fenomeni:
\begin{itemize}
\item L'assorbimento della radiazione di tipo esponenziale con il coefficiente di assorbimento (nota anche come legge di Lambert-Beer): $\exp{\left( -\int_{\vec{r_0}}^{\vec{r'}} \mu(\vec{r'},E) \de l \right)}$.
\item La divergenza del fascio che da luogo ad una decrescita della fluenza con l'inverso del quadrato della distanza ($|\vec{r_0}|^2 / |\vec{r'}|^2$).
\end{itemize}
\end{itemize}
L'equazione che si ottiene per la fluenza riferita ad un qualsiasi punto di coordinata $\vec{r'}$ all'interno del paziente è:
\begin{equation}
\label{eq:fluence}
\Psi(\vec{r'},E) = \frac{|\vec{r_0}|^2}{|\vec{r'}|^2}\Psi(\vec{r_0},E)\,\exp{\left( -\int_{\vec{r_0}}^{\vec{r'}} \mu(\vec{r'},E) \de l \right)}
\end{equation}
Lo studio CT del paziente viene discretizzato nel TPS su una griglia di voxel cubici detta \textit{dose grid} (su cui poi verrà calcolata la distribuzione di dose). I voxel della dose-grid campionano le densità fornite dalla CT da cui viene ricavato il coefficiente di assorbimento locale $\mu / \rho(\vec{r'},E)$.\\
Moltiplicando il termine $\mu / \rho(\vec{r'},E)$  con la fluenza espressa dalla \eqref{eq:fluence} si ottiene la distribuzione di TERMA all'interno del paziente (Eq.\eqref{eq:termaE}).



\subsection{Applicazione delle funzioni di scatter}
\label{sec:scatter_fun}

\begin{figure}
\centering
\includegraphics[width=.7\textwidth]{./cap1/compt_dom.png}
\caption{Prevalenza degli effetti di interazione dei fotoni con la materia in funzione dell'energia del fotone. L'effetto Compton è prevalente per le energie e i materiali tipici della radioterapia.}
\label{fig:compt_dom}
\end{figure}
Alle energie tipiche dei fotoni utilizzati in radioterapia l'effetto predominante è l'effetto Compton (Fig.\ref{fig:compt_dom}) secondo cui l'interazione del fotone primario genera un fotone diffuso ed un elettrone con una certa energia cinetica. Una modellizzazione completa del trasporto del fotone diffuso e dell'elettrone messo in moto richiede un approccio statistico di tipo Monte Carlo, computazionalmente molto dispendioso. L'approccio convolution/superposition racchiude i processi statistici di deposizione dell'energia nella funzione di scatter introdotta nell'Eq.\eqref{eq:superp} ed agisce in maniera deterministica.

La funzione di scatter è ottenibile a partire da una simulazione Monte Carlo in cui un unico fotone viene fatto interagire forzatamente all'interno di un volume omogeneo di acqua (elemento più simile ai tessuti molli umani) e si va ad osservare la deposizione della dose che ne deriva. Questa operazione è stata effettuata da Mackie \cite{Mackie1985} utilizzando il software Monte Carlo EGS sviluppato dallo Stanford Linear Accelerator Center che pubblicò nella metà degli anni '80 delle funzioni di scatter discretizzate in forma tabulare (vedi Fig.\ref{fig:mackie_kernels}).
\begin{figure}
\centering
\includegraphics[width=.8\textwidth]{./cap1/mackie_kernels.png}
\caption{Funzione di scatter calcolate per un fotone da 15MV e per due differenti materiali di densità  1g/cc (acqua) e 0.2g/cc. \cite{Mackie1985}}
\label{fig:mackie_kernels}
\end{figure}

Nota la funzione di scatter, assieme alla distribuzione di TERMA, è necessario procedere alla risoluzione dell'integrale indicato nell'Eq.\eqref{eq:superp} per arrivare alla distribuzione di dose assorbita. 
Questa operazione è effettuata per via numerica (tramite computer) e presuppone una discretizzazione dello spazio di integrazione e delle funzioni integrande. I metodi adottabili sono i seguenti:\\

\textbf{Somma diretta:\\}
Applicando le funzioni di scatter (kernel) ai punti di deposizione del TERMA, in linea di principio è possibile arrivare alla dose assorbita in maniera triviale risolvendo per somma diretta l'integrale \eqref{eq:superp}. Tuttavia, questa operazione su una griglia di $N^3$ voxel e generalizzata ad un mezzo disomogeneo, risulta essere un problema  di ordine $N^7$ \cite{Ahnesjo1989}. Un tale numero di operazioni è difficilmente gestibile anche con le moderne potenze di calcolo\footnote{Se immaginiamo di discretizzare un paziente di dimensioni $20x20x20$ cm$^3$ con voxel di lato $0.3$ cm otteniamo una griglia di circa $70x70x70$ voxel. Il numero di operazioni da effettuare risolvendo numericamente l'integrale \eqref{eq:superp} sarebbe $70^7\approx 10^{13}$. Disponendo di un computer capace di effettuare operazioni con frequenza dell'ordine del GHz$\equiv 10^9\,s^{-1}$ il tempo di risolvere $10^{13}$ operazioni sarebbe $t=10^{13}/10^9\,s=10^5\,s\approx 2\, ore!$.}. \\

\textbf{Metodo di Fast Fourier Transform (FFT):\\}
Un possibile approccio alternativo alla risoluzione della \eqref{eq:superp} si basa sulla considerazione per cui, nel caso di un mezzo omogeneo, la funzione di scatter della \eqref{eq:superp} risulta essere spazialmente invariante (l'integrale è un prodotto di convoluzione). Questo permette l'applicazione della teoria degli spazi di Fourier per cui l'integrale di convoluzione  diventa un semplice prodotto delle trasformate delle funzioni nello spazio di Fourier. L'operazione di antitrasformata del prodotto delle funzioni trasformate permette di ottenere la distribuzione di dose. Questa operazione è stata dimostrata comportare un numero di operazioni inferiore alla somma diretta pari a $N^3\log_2 N$ \cite{Wong1996}. Il limite di questa metodologia è che risulta esattamente applicabile solo su uno spazio continuo. La discretizzazione che inevitabilmente va effettuata per un calcolo numerico comporta delle approssimazioni. Inoltre questo metodo risulta essere difficilmente estensibile al caso disomogeneo in cui il kernel perde di invarianza spaziale. Sono stati studiati vari approcci basati su correzioni orientate a recuperare l'invarianza spaziale dei kernel anche nel caso disomogeneo. I risultati tuttavia non si sono rivelati soddisfacenti se comparati con il metodo presentato nel paragrafo successivo \cite{Wong1996}.\\

\textbf{Metodo di \textit{Collapsed-cone-convolution}:\\}
Questo metodo è quello che si è rivelato il più soddisfacente nella storia degli algoritmi \textit{model-based} in quanto facilmente estensibile al caso disomogeneo e polienergetico. Contiene inoltre come parte fondante una discretizzazione spaziale dei kernel di deposizione facilmente adattabile ad un calcolo numerico tramite computer. Il metodo \textit{collapsed-cone} è alla base del motore dosimetrico implementato nel TPS RayStation e verrà presentato nel dettaglio nella sezione successiva.


\subsection{L'approssimazione \textit{collapsed-cone}}
L'algoritmo collapsed-cone discretizza spazialmente i kernel di deposizione prima di effettuare la convoluzione in modo da tener conto del trasporto della radiazione non in tutte le direzioni ma solo in un loro sottoinsieme (questa operazione avverrebbe in ogni caso impiegando un qualsiasi metodo di risoluzione numerico tramite computer).

Questo approccio è stato presentato indipendemente da Mackie \cite{Reckwerdt1992} e Ahnesj\"{o} \cite{Ahnesjo1989} e fu denominato \textit{collapsed-cone}. Il confronto della dose ottenuta con questo metodo ed il metodo Monte Carlo fornì risultati eccellenti per l'epoca. Questi risultati rappresentano ancora oggi il benchmark per il calcolo della dose con TPS utilizzando metodi non statistici.
\begin{figure}[!t]
\centering
a) \includegraphics[width=.42\textwidth]{./cap1/kern_ray1.png}$\quad$
b) \includegraphics[width=.42\textwidth]{./cap1/kern_ray2.png}
\caption{(a) distribuzione planare di un kernel di deposizione. (b) distribuzione 1D di alcuni kernel al variare dell'energia; da notare una prima parte di rilascio rapido di energia dovuta agli elettroni Compton ed una parte più estesa dovuta ai fotoni secondari diffusi che a loro volta re-interagiscono con il mezzo.}
\label{fig:kern_ray}
\end{figure}

Per capire il meccanismo alla base di questa approssimazione partiamo con l'osservare in Fig.\ref{fig:kern_ray} la distribuzione spaziale tipica di alcuni kernel di deposizione. La discretizzazione di questi kernel può essere effettuata in maniera diretta lungo delle semplici linee di integrazione seguendo un processo noto come \textit{ray-tracing}.
\begin{figure}
\centering
a)\includegraphics[width=.4\textwidth]{./cap1/Kernel_RayTr.eps}
b)\includegraphics[width=.45\textwidth]{./cap1/Kernel_CCC.eps}
\caption{(a) Errore di sampling che si commette discretizzando il kernel solo lungo dei raggi (Ray-tracing). (b) Meccanismo di ridistribuzione della dose in un settore conico sul proprio asse (Collapsed-cone).}
\label{fig:raytrace_vs_cc}
\end{figure}
Tuttavia Ahnesj\"{o} notò che questo metodo di discretizzazione semplicistico può portare ad un errore di sampling ampio a causa sia della divergenza dei raggi di ray-tracing sia della particolare forma dei kernel di deposizione  \cite{Ahnesjo1989} (i.e. una larga parte di deposizione di energia non viene conteggiata a meno di usare un campionamento angolare fittissimo complicato da gestire a livello computazionale (vedi Fig.\ref{fig:raytrace_vs_cc}a)).\\
Per questo Ahnesj\"{o} propose di discretizzare i kernel in coordinate sferiche con dei coni aventi il vertice nel punto di interazione e sottendenti un certo angolo solido. Tutte le deposizioni di energia contenute all'interno della superficie conica che sottende l'angolo solido vengono \textquotedblleft\textit{collassate}\textquotedblright{} sull'asse del cono stesso (vedi Fig.\ref{fig:raytrace_vs_cc}b). Con questa procedura, l'energia totale depositata risulta spazialmente ridistribuita ma viene conteggiata per intero.

L'operazione di \textquotedblleft\textit{collassamento}\textquotedblright{} dei kernel avviene matematicamente come segue. In primo luogo si effettua un fit analitico del kernel di deposizione calcolato tramite metodologie Monte Carlo. La funzione proposta da Ahnesj\"{o} per questo scopo in coordinate sferiche ha la seguente forma:
\begin{equation}
\label{eq:kern_fit}
s(r,\theta) = \frac{A_\theta e^{-a_\theta r} + B_\theta e^{-b_\theta r}}{r^2}
\end{equation}
dove $A_\theta,\,a_\theta,\,B_\theta$ e $b_\theta$ sono parametri di fit che dipendono dall'angolo di scatter $\theta$. I due termini esponenziali descrivono l'uno la caduta rapida del kernel dovuto agli elettroni Compton primari e l'altro la coda più lenta dovuta alle interazioni di scatter secondarie (Fig.\ref{fig:kern_ray}b). Da notare che il kernel possiede una simmetria cilindrica per cui l'angolo $\phi$ della tripletta di coordinate sferiche ($r,\theta,\phi$) non è esplicitato. La validità di questo fit è mostrata in Fig.\ref{fig:kern_fit}.
\begin{figure}
\centering
\includegraphics[width=.8\textwidth]{./cap1/kern_fit.png}
\caption{Fit analitico dei kernel di deposizione per un fotone da 6MV (a) e da 15MV (b) \cite{Ahnesjo1989}.}
\label{fig:kern_fit}
\end{figure}

Stabilendo un angolo solido $\Omega_i$, l'operazione di \virg{collassamento} è un'integrazione in coordinate sferiche del kernel espresso dalla \eqref{eq:kern_fit} (vedi anche Fig.\ref{fig:kern_collaps}):
\begin{equation}
\iint_{\Omega_i} s(r,\theta) r^2\de \Omega_i = A_{\Omega_i} e^{-a_{\Omega_i} r} + B_{\Omega_i} e^{-b_{\Omega_i} r} \equiv K(r,\Omega_i)
\end{equation}
La funzione $K(r,\Omega_i)$ è quella da convolvere con la mappa del TERMA per arrivare alla mappa di dose assorbita.

Il metodo collapsed-cone esteso al caso più generico (fascio polienergetico e mezzo disomogeneo) è stato dimostrato comportare un numero di operazioni dell'ordine di $MN^3$ dove $M$ è il numero di settori in cui viene diviso l'angolo solido ed $N$ è il numero di voxel della griglia di dose/TERMA \cite{Ahnesjo1989}. Questo risultato è da confrontare con l'ordine $N^7$ in caso di somma diretta di tutti i termini che si ottengono discretizzando l'integrale \eqref{eq:superp} oppure con $N^3\log_2N$ nel caso di applicazione del metodo di trasformata di Fourier.

\begin{figure}
\centering
\includegraphics[width=\textwidth]{./cap1/kern_collaps.png}
\caption{Operazione di \textit{collapsing} dei kernel all'interno di segmenti conici definiti dall'angolo solido $\Omega_i$.}
\label{fig:kern_collaps}
\end{figure}

In RayStation il numero di settori in cui viene discretizzato il kernel è 128 che corrisponde a 8 direzioni lungo l'angolo di scatter $\theta$ e 16 lungo l'angolo di simmetria cilindrica $\phi$. Il campionamento è più fitto nella direzione di propagazione del fascio dove è presente il gradiente più intenso.

Così come implementato il metodo non richiede nelle ipotesi che il kernel sia un invariante spaziale come nel caso del metodo a trasformata di Fourier. Questo rende semplice dal punto di vista computazionale  la generalizzazione al caso disomogeneo che verrà trattata in seguito alla generalizzazione al caso polienergetico nelle sezioni a seguire.

\begin{figure}
\centering
\includegraphics[width=.7\textwidth]{./cap1/LinacSpectrum.png}
\caption{Spettro di energia tipico di un fascio radioterapico di energia 6MV (solo componente fotonica)}
\label{fig:LinacSpectrum}
\end{figure}

\subsection{La generalizzazione al caso polienergetico}
Un tipico fascio di fotoni generato da un LINAC per radioterapia contiene uno spettro polienergetico (vedi Fig.\ref{fig:LinacSpectrum}). Inoltre lo spettro subisce uno spostamento verso energie più alte a causa dell'attraversamento della materia (effetto di \textit{depth-hardening}). Questo suggerisce il fatto che non è possibile considerare un unico kernel di deposizione nel passaggio dal TERMA alla dose. Boyer et al. assieme a Zhu e Van Dyk \cite{Boyer1989,Zhu1995} hanno dimostrato che una discretizzazione in 5 bin sia sufficiente a rappresentare lo spettro di un LINAC da 6MV. Sulla base di ciò, Papanikolaou et al. \cite{Papanikolaou1993} hanno investigato i possibili metodi di estensione del metodo convolution al caso polienergetico dimostrando che è sufficiente combinare kernel monoergetici pesati con le relative componenti spettrali. Nel medesimo lavoro viene dimostrato che il problema del depth-hardening può essere risolto riscalando i kernel con dei fattori che dipendono dalla lunghezza radiologica attraversata.

Per applicare questi principi, nel TPS sono memorizzati una serie di point spread kernel computati con il motore Monte Carlo EDKnrc incluso in EGS4nrc per le seguenti energie del fotone primario incidente: 0.5, 1, 1.5, 2, 2.5, 3, 3.5, 4, 5, 6, 7, 8, 9 10, 12, 14, 16, 18, 20 MeV. Queste energie corrispondono ai bin implementati in RayStation per rappresentare uno spettro fotonico (es. per un fascio da 6 MV i bin coinvolti vanno da 0.5 a 6 MV).\\
Ad ogni bin dello spettro corrisponde un singolo kernel monoenergetico. Il kernel polienergetico viene calcolato mediante media dei singoli kernel monoenergetici pesati con l'intensità delle rispettive componenti spettrali  \cite{Papanikolaou1993}. Questo processo è raffigurato schematicamente nella Fig.\ref{fig:kern_trans}a.
In seguito viene costruita una libreria di 600 kernel riscalati su un ampio intervallo di lunghezze radiologiche per tenere conto dell'effetto di \textit{depth-hardening} nelle varie situazioni cliniche.


\subsection{La generalizzazione al caso disomogeneo}

Considerando un mezzo con varie disomogeneità, la funzione di scatter perde la sua invarianza spaziale per cui l'equazione \eqref{eq:superp} cessa di essere un'integrale di convoluzione. In termini fisici, in un mezzo disomogeneo non sussiste più l'equilibrio di particelle cariche in ogni punto del volume.

In condizioni standard i kernel sono calcolati in un volume omogeneo di acqua, tuttavia la deposizione di energia cambia se i processi di ionizzazione avvengono in un mezzo differente. Per questo motivo, in linea di principio, i kernel di deposizione andrebbero di volta in volta ricalcolati a seconda della situazione.\\
Mackie propose un'approssimazione orientata ad evitare di ricalcolare i kernel in ogni singola situazione clinica. Secondo questa approssimazione, il kernel di deposizione per un mezzo disomogeneo si può ottenere a partire dal kernel calcolato in acqua e sostituendo tutte le lunghezze fisiche ($l$) con le lunghezze radiologiche ($\rho_{H_2O}^{e^-}\cdot\,l$) (dove $\rho_{H_2O}^{e^-}$ è la densità elettronica del mezzo relativa alla densità elettronica dell'acqua\footnote{\`E necessario utilizzare la densità elettronica in quanto è tra gli elettroni del mezzo che avvengono le interazioni che portano all'assorbimento della dose.}) \cite{Mackie1985}. Questa operazione è nota con il nome di \textit{density-scaling} dei kernel di deposizione ed era in realtà conosciuta già prima dell'avvento degli algoritmi di convolution come Teorema di O'Connor \cite{OConnor1957}.\\
\begin{figure}[!t]
\centering
a) \includegraphics[width=.36\textwidth]{./cap1/kern_dens.png}
b)\includegraphics[width=.5\textwidth]{./cap1/kern_dens2.png}
\caption{(a) Confronto tra un kernel calcolato direttamente con un metodo Monte Carlo (linea solida) e un kernel calcolato in condizioni omogenee riscalato secondo la lunghezza radiologica (linea tratteggiata); le isolinee corrispondono ad una dose relativa di 1000, 500, 100, 50, 10, 5, 1; il punto A posizionato appena sotto l'interfaccia aria/acqua presenta la massima discrepanza (50\%). (b) effetto globale su un insieme di interazioni; un \textit{averaging} delle incertezze sui singoli kernel si risolve in un accordo accettabile su un fantoccio complesso che simula un torace \cite{Woo1990,Arnfield2000,Ahnesjo1989}.}
\label{fig:kern_dens}
\end{figure}
In particolare questo teorema fu formulato per mettere in relazione la dose calcolata in due mezzi di differente densità ma stessa composizione chimica. In queste condizioni O'Connor dimostrò che il numero di fotoni scatterati rispetto al numero di fotoni primari rimane invariato nei due mezzi qualora tutte le distanze geometriche (includendo anche la dimensione del campo, la distanza dalla sorgente etc.) siano riscalate con le lunghezze radiologiche \cite{OConnor1957}. Le ipotesi alla base di questo teorema sono:
\begin{itemize}
\item Mezzo infinitamente esteso (per cui si realizza la condizione di equilibrio di particelle cariche).
\item Composizione chimica dei due mezzi identica.
\end{itemize}
Queste condizioni non sono realizzabili in pratica per cui è inevitabile che il metodo del density-scaling dei kernel comporti degli errori quando applicato ad una situazione clinica reale.

Nella Fig.\ref{fig:kern_dens}a si può notare l'errore che si commette applicando il teorema di O'Connor nella forma suggerita da Mackie ai kernel di deposizione in una configurazione di disomogeneità aria/acqua che può simulare una situazione tessuto/polmone (Woo e Cunningham \cite{Woo1990}). Nonostante la comparazione del kernel calcolato direttamente e del kernel calcolato con il metodo del density-scaling presenti degli errori fino al 50\%, è stato dimostrato \cite{Ahnesjo1989} che su un insieme numeroso di interazioni interviene un effetto di \textit{averaging} per cui discrepanze significative rimangono solo in condizioni di estrema disomogeneità, ossia vicino alle interfacce in cui si ha un cambio repentino di densità (vedi Fig.\ref{fig:kern_dens}b).

%L'integrale espresso nell'Eq.\ref{eq:superp} può essere riscritto per un mezzo omogeneo nella forma \cite{Khan2010}:
%\begin{equation}
%D(x,y,z) =  \int_V \frac{\mu}{\rho} \Psi(x',y',z')\,s(x'- x, y'- y, z'- z)\, \de x' \de y' \de z'
%\label{eq:superp_homo}
%\end{equation}
%Questa equazione rappresenta \textit{principio di convolution} per il calcolo della dose. Considerando un mezzo infinitamente esteso, la funzione $s$ è spazialmente invariante. Si è già accennato nella Sez.\ref{}

In Fig.\ref{fig:kern_trans}b è riportato un diagramma schematico della modificazione di un kernel in un mezzo disomogeneo quando viene applicato il metodo del \textit{density-scaling}.

Applicando il teorema del \textit{density-scaling} (o di O'Connor) alla \eqref{eq:superp}, sostituendo le lunghezze fisiche con le lunghezze radiologiche \cite{Khan2010} si ottiene la seguente equazione:
\begin{equation}
D(\vec{r}) = \iint_{E,V} T_E(\rho_{\vec{r'}} \cdot \vec{r})\, s_E(\rho_{\vec{r}-\vec{r'}}\cdot (\vec{r}-\vec{r'}))\de\vec{r'} \de E
\label{eq:superp_disom}
\end{equation}
dove $(\rho_{\vec{r'}} \cdot \vec{r})$ è la lunghezza radiologica dalla sorgente di fotoni fino al punto di interazione primaria (ove viene rilasciato il TERMA) e $\rho_{\vec{r}-\vec{r'}}\cdot (\vec{r}-\vec{r'})$ è la lunghezza radiologica dal punto di interazione primaria al punto di deposizione della dose.\\
Questa equazione è nota come equazione di \textit{convolution-superposition} e rappresenta il problema generico del calcolo della dose per un fascio polienergetico e mezzo disomogeneo. Essa è alla base del motore dosimetrico implementato in RayStation ed è risolta numericamente utilizzando il metodo collapsed-cone.

\begin{figure}
\centering
\includegraphics[width=\textwidth]{./cap1/kern_trans.png}
\caption{(a) figura schematica del metodo con cui si costruisce il kernel polienergetico (combinazione di kernel monoenergetici pesati con la componente spettrale). (b) figura schematica che rappresenta come il kernel si modifica quando viene riscalato tenendo conto delle lunghezze radiologiche attraversate nelle varie direzioni.}
\label{fig:kern_trans}
\end{figure}




\section{Approssimazioni aggiuntive}
Nella sezione precedente sono state mostrate le approssimazioni intrinseche dell'algoritmo \textit{collapsed-cone}. In questa sezione evidenziamo delle ulteriori approssimazioni adottate in RayStation orientate all'aumento della velocità di calcolo della dose.

\subsection{L'approssimazione \textit{no-kernel-tilting}}
Un fascio di fotoni generato da un LINAC è tipicamente divergente all'aumentare della distanza dalla sorgente. I kernel di deposizione dell'energia sono calcolati per un fascio che incide verticalmente ed hanno l'asse orientato nella medesima direzione. Un'applicazione rigorosa del metodo convolution/superposition richiederebbe la rotazione dei kernel per allinearli lungo l'angolo di incidenza del raggio primario (vedi Fig.\ref{fig:kern_tilt}a).\\
\begin{figure}
\centering
a)\includegraphics[width=.3\textwidth]{./cap1/kern_tilt.png}
b)\includegraphics[width=.55\textwidth]{./cap1/kern_tilt_b.png}
\caption{(a) I kernel di deposizione dovrebbero essere ruotati dello stesso angolo di incidenza $\nu$ del raggio primario; tuttavia per ridurre considerevolmente i tempi di calcolo, si applicano kernel con orientazione verticale. (b) il non ruotare i kernel comporta una deposizione di dose maggiore verso il centro del fascio (linea tratteggiata) confrontata con la dose calcolata senza approssimazione (linea solida).}
\label{fig:kern_tilt}
\end{figure}
Questa operazione implicherebbe molteplici rotazioni di matrici \cite{Sharpe1997} con il risultato di aumentare considerevolmente il tempo di calcolo della dose. 
%Oltre a questo, un ulteriore grande vantaggio dell'approssimazione \textit{no-kernel-tilting} consiste nel fatto di poter calcolare solo una volta le intersezioni dei raggi uscenti dai kernel con la griglia di dose\footnote{Operazione nota come \textit{raytracing}.} e di riutilizzare i risultati per i voxel successivi.
Per questo motivo vari autori hanno sviluppato dei metodi che consentissero di applicare kernel di deposizione non ruotati (\textit{no-kernel-tilting}) \cite{Sharpe1997,Papanikolaou1993}.

In RayStation è implementato il metodo suggerito da Papanikolaou et al. \cite{Papanikolaou1993}. L'idea di questo metodo sta nel considerare che, non ruotando i kernel, si ha un accumulo di dose maggiore verso il centro del fascio (Fig.\ref{fig:kern_tilt}b).\\ 
Papanikolaou ha dimostrato come questo può essere recuperato con i seguenti passaggi:
\begin{enumerate}
\item Si rimuove la divergenza dalla fluenza di energia (termine $|\vec{r_0}|^2/|\vec{r'}|^2$ nell'Eq.\eqref{eq:fluence}).
\item Si applicano i point spread kernel con l'approssimazione di \textit{no-tilting}.
\item Si riapplica la divergenza riscalando la dose calcolata con il fattore $|\vec{r_0}|^2/|\vec{r}|^2$ dove $\vec{r}$ è la coordinata di calcolo della dose (diversa dalla coordinata di rilascio del TERMA $\vec{r'}$).
\end{enumerate}
L'effetto dell'operazione n.1 è quello di aumentare la fluenza di energia lontano dal centro del fascio. Applicando poi i punti 2 e 3 si arriva ad una distribuzione in cui la dose approssima quella calcolata considerando i kernel ruotati.\\
Sharpe e Papanikolaou \cite{Sharpe1997,Papanikolaou1993} hanno evidenziato come l'approssimazione di \textit{no-tilting} dei kernel presenti i suoi limiti solo in situazioni difficilmente realizzabili clinicamente. In particolare, discrepanze oltre il 3\% si osservano in caso di basse distanze sorgente-superficie del paziente (SSD $< 70$ cm) congiuntamente all'utilizzo di campi estesi ($> 20x20$ cm$^2$).

\begin{figure}
\centering
\includegraphics[width=.6\textwidth]{./cap1/TERMADepos.eps}
\caption{Distinzione delle zone utilizzate nell'approssimazione \textit{full-TERMA-deposition}. (CCC: Collapsed Cone Convolution)}
\label{fig:TERMADepos}
\end{figure}

\subsection{Approssimazioni \textit{full-TERMA-deposition} e \textit{adaptive interpolation}}
Ulteriori approssimazioni denominate \textit{full-TERMA-deposition} e \textit{adaptive interpolation} sono volte ad evitare di applicare rigorosamente il metodo di collapsed-cone-superposition su ogni voxel della griglia di dose.\\
A questo proposito, la \textit{full-TERMA-deposition} può essere riassunta in quattro principali step:
\begin{enumerate}
\item Alla fine del calcolo della distribuzione di TERMA vengono identificati quei voxel in cui è stato depositato un TERMA minore dello 0.5\% del massimo TERMA rilasciato nel volume (zona (1) nella Fig.\ref{fig:TERMADepos}).
\item A partire da questi voxel viene costruita una iso-superficie che viene poi espansa di una lunghezza radiologica di 5 cm isotropicamente.
\item All'interno del volume compreso tra la superficie TERMA$_i= 0.5\%$ TERMA$_{max}$ e la sua espansione di $5$ cm  il trasporto avviene solo su alcune delle 128 direzioni del kernel di deposizione (zona (2) nella Fig.\ref{fig:TERMADepos}). 
\item Al di fuori della superficie precedentemente calcolata non viene effettuato il trasporto del TERMA con l'algoritmo collapsed-cone bensì si assume un rilascio locale di tutta l'energia (zona (3) nella Fig.\ref{fig:TERMADepos}).
\end{enumerate}
Questa approssimazione è stata osservata generare un'errore sul calcolo della dose minore dello 0.2\% su punti in generale a bassa rilevanza clinica \cite{RaySearchLaboratories2014}.\\

\begin{figure}[!t]
\centering
\includegraphics[width=.8\textwidth]{./cap1/dose_interp.png}
\caption{L'approssimazione di \textit{adaptive interpolation}. I voxel in blu sono quelli calcolati con l'algoritmo collapsed-cone.}
\label{fig:dose_interp}
\end{figure}

La seconda approssimazione di \textit{adaptive interpolation} si basa su un preliminare calcolo della dose su voxel alternati per identificare zone a basso od alto gradiente. Per le zone a basso gradiente la dose nei voxel non coinvolti nel computo preliminare viene calcolata con la media dei voxel adiacenti; per le zone ad alto gradiente viene applicato l'algoritmo completo. In particolare gli step che vengono seguiti sono riassumibili nei seguenti punti:
\begin{enumerate}
\item Dopo il calcolo della distribuzione di TERMA si effettua un primo calcolo della dose ogni due voxel come illustrato nella Fig.\ref{fig:dose_interp}.
\item Un voxel viene considerato appartenere ad una zona a basso gradiente se la massima differenza tra il TERMA nei voxel adiacenti è minore dello 0.2\% e la massima differenza in dose è minore del 5\% della dose massima.
\item Per i voxel appartenenti alla zona a basso gradiente la dose viene calcolata mediando i voxel adiacenti che possono essere in numero di 2, 4 o 8 a seconda della posizione (vedi Fig.\ref{fig:dose_interp}).
\item Per i voxel appartenenti alla zona ad alto gradiente viene applicato l'algoritmo completo.
\end{enumerate}
Per un semplice campo $10x10$ cm$^2$ ed energia 6 MV, la distribuzione di dose calcolata senza applicare la \textit{adaptive interpolation} si correla con la dose calcolata utilizzando l'approssimazione con un coefficiente di correlazione maggiore di 0.99999. Gli errori più rilevanti si registrano vicino alla penombra del fascio dove rimangono comunque entro il 2\% \cite{RaySearchLaboratories2014}.


\section{Dose aggiuntiva da elettroni di contaminazione}
\label{sec:dose_electr}
Come già accennato nella sezione introduttiva (Fig.\ref{fig:processes}), nei primi centimetri di tessuto una parte rilevante della dose totale è dovuta agli elettroni di contaminazione del fascio fotonico. Questi elettroni si originano prevalentemente da processi di scatter dei fotoni con i collimatori ed il flattening-filter all'interno della testata.\\
Alle energie tipiche della radioterapia, la deposizione della dose di un fascio elettronico avviene repentinamente all'aumentare della profondità (vedi Fig.\ref{fig:electr_enloss}a).
\begin{figure}
\centering
a) \includegraphics[width=.45\textwidth]{./cap1/electr_enloss.png}
b) \includegraphics[width=.44\textwidth]{./cap1/electr_doseCT_zoom.png}
\caption{(a) diagrammi della dose rilasciata in funzione della profondità di un fascio di elettroni a varie energie. (b) distribuzione di dose dovuta ad elettroni di contaminazione del fascio calcolata su una CT. Da notare è che questa dose risulta importante solo nei primi millimetri di tessuto attraversato.}
\label{fig:electr_enloss}
\end{figure}

L'algoritmo dosimetrico implementato in RayStation per il calcolo della dose da contaminazione elettronica rientra nella classe degli algoritmi \textit{model-based} basati su \textit{convolution/superposition}, con una variante sul kernel di deposizione che non è più puntuale ma esteso di forma cilindrica (vedi Fig.\ref{fig:electr_pencil}). 

\begin{figure}
\centering
a) \includegraphics[width=.44\textwidth]{./cap1/electr_pencil.png}$\qquad$\\
b) \includegraphics[width=.44\textwidth]{./cap1/electr_pencil_b.png}
\caption{(a) kernel di deposizione di tipo \textit{pencil-beam}. (b) figura schematica che mostra la simmetria cilindrica nel calcolo della dose quando si utilizza un kernel di tipo \textit{pencil-beam}.}
\label{fig:electr_pencil}
\end{figure}


Questo tipo di algoritmo è denominato \textit{pencil-beam} e rappresenta una versione semplificata del collapsed-cone.\\
L'equazione che permette il calcolo della dose è analoga alla \eqref{eq:superp} rappresentabile in coordinate cilindiriche:
\begin{equation}
\label{eq:electr_superp}
D(d) = 2\pi \iint_{E,V} \Psi(r,d,E)\, s(r,d,E)\, r\de r \de E
\end{equation}
dove $E$ e $V$ sono rispettivamente l'energia ed il volume di integrazione, $r$ la coordinata di integrazione e $d$ il punto di valutazione della dose.
L'Eq.\eqref{eq:electr_superp} è implementata all'interno di RayStation in maniera analoga al caso dei fotoni. Schematicamente i passaggi che vengono seguiti sono:
\begin{itemize}
\item I kernel \textit{pencil-beam} sono pre-calcolati con un modello Monte Carlo (software EGSnrc).
\item Viene utilizzato un modello a due sorgenti per calcolare la fluenza di energia elettronica in assenza del paziente.
\item Viene calcolata la distribuzione di TERMA utilizzando la formula della perdita di energia per ionizzazione di Bethe-Bloch per gli elettroni nella materia \cite{RaySearchLaboratories2014}.
\item Vengono applicati i kernel \textit{pencil-beam} per modellizzare lo scatter laterale degli elettroni.
\end{itemize}

Nella Fig.\ref{fig:electr_enloss}b è mostrata la distribuzione di dose dovuta agli elettroni di contaminazione calcolata su una CT di un paziente. Questa dose viene sommata alla dose dovuta ai fotoni calcolata con l'algoritmo collapsed-cone. 
















\chapter{Lo Scanner NMR di Superficie}
\minitoc
\textsf{In questo secondo capitolo verranno discusse...}



\section{Conclusioni}
In questo capitolo è stato descritto dettagliatamente ...





\chapter{La radioterapia adattiva. Implementazione iniziale e sviluppi futuri}
\minitoc
\textsf{In questo capitolo si passa ....}

\section{Origini e razionale della radioterapia adattiva}
Per radioterapia adattiva o \textit{adaptive radiation therapy} si intende una tecnica di erogazione della dose che di `adatta' ai possibili cambiamenti che si verificano durante il trattamento, rispetto alla situazione iniziale di pianificazione. I cambiamenti che si possono verificare durante un trattamento radiante possono essere di varia natura e riguardare sia l'interno del paziente (progressione/regressione del target, dimagrimento, diverso stato di riempimento degli organi cavi), sia il suo  posizionamento sul lettino di trattamento (incertezza di setup). Il metodo tradizionale di affrontare il problema dei cambiamenti rispetto alla situazione di pianificazione è esposto nei report della \textit{International Commission of Radiological Units} ICRU no.50 e no.62, ripreso anche nel recente report no.83 per le tecniche ad intensità modulata \cite{ICRU2010}. Esso consiste nell'espandere i target di un certo margine di sicurezza in modo da assicurare la loro corretta irradiazione anche in presenza di incertezze dovute al movimento degli organi interni e al posizionamento del paziente.
\begin{figure}
\centering
\includegraphics[width=.3\textwidth]{./cap3/ptv.png}
\includegraphics[width=.69\textwidth]{./cap3/adapt0.png}
\caption{Sinistra: definizioni ICRU dei volumi coinvolti nella pianificazione del trattamento radioterapico. Destra: Processo standard di pianificazione ed erogazione del piano del trattamento.}
\label{fig:adapt0}
\end{figure}

Nella Fig.\ref{fig:adapt0} sono riportate le definizioni dei volumi coinvolti nella pianificazione del trattamento radioterapico assieme ad uno schema del processo standard di pianificazione ed erogazione del piano del trattamento.\\
Il meccanismo che ha portato alla definizione dei vari volumi in radioterapia procede in vari step:
\begin{description}
\item[Gross Target Volume (GTV):] questo volume identifica il tumore a livello macroscopico (rivelabile attraverso strumenti ottici, imaging radiologico ed evenutale esame obiettivo del medico).

\item[Clinical Target Volume (CTV):] volume che comprende il GTV ed un margine attorno ad esso in cui vi è alta probabilità di presenza di malattia che ancora non si manifesta clinicamente (sub-clinica).

\item[Internal Target Volume (ITV):] volume che aggiunge al CTV un margine (detto \textit{internal margin o IM}) che serve a comprendere il possibile movimento del target per cause interne al paziente (e.g. respiro, stato di riempimento degli organi adiacenti, battito cardiaco\ldots).

\item[Planning Target Volume (PTV):] volume che aggiunge all'ITV un margine per comprendere l'incertezza dovuta al posizionamento del paziente rispetto alla posizione di pianificazione (\textit{setup margin o SM}).

\item[Treated Volume:] volume che riceve una dose maggiore del 98\% della dose di prescrizione.

\end{description}

Il metodo per calcolare i margini necessari alla definizione del PTV si basa su uno studio statistico in cui si effettua un analisi dei possibili errori di posizionamento del target rispetto alla posizione pianificata, distinguendoli in errori sistematici ed errori random. Una delle formule più note è dovuta a Marcel Van Herk \cite{ICRU62}:
\begin{equation}
PTV_{margin} = 2.5\Sigma + 0.7\sigma
\end{equation}
dove $\Sigma$ rappresenta l'errore sistematico ossia quell'errore che si ripete per tutto il ciclo di trattamento (es. paziente dimagrito rispetto alla pianficazione) mentre $\sigma$ è l'errore random (es. movimento d'organo).

In un approccio non adattivo i margini da dare al CTV per arrivare al PTV si trovano effettuando uno studio di coorte per un fissato sito di irradiazione. Nello studio si valutano gli errori sistematici e random su una popolazione di pazienti e si applica la formula di Van Herk secondo le direttive ICRU \cite{ICRU62}. La pianificazione viene quindi effettuata stabilendo delle condizioni di irradiazione del planning target volume (PTV)  che vengono ripetute per tutto il ciclo di trattamento (Fig.\ref{fig:adapt0}).\\
Questo tipo di approccio può tuttavia essere non è ottimale per lo specifico paziente come indicato schematicamente nella Fig.\ref{fig:margins}. In un approccio di \textit{adaptive radiotherapy} è invece possibile studiare il singolo caso ed adattare il trattamento allo specifico paziente.
\begin{figure}
\centering
\includegraphics[width=.8\textwidth]{./cap3/margins.png}
\caption{La formula di Van Herk per ottenere il PTV utilizzando uno studio di coorte (sinistra) da luogo a margini che possono essere sovrabbondanti per il singolo paziente (destra).}
\label{fig:margins}
\end{figure}

Il termine \textit{adaptive radiotherapy} è stato introdotto per la prima volta  da Yan et al.\cite{Yan1996}. In questo lavoro veniva per la prima volta proposto un adattamento del piano di cura a seguito di rilevazioni effettuate nelle prime cinque sedute di trattamento con un dispositivo in grado di `fotografare' la posizione del paziente e di confrontarla rispetto alla posizione di pianficazione\footnote{Il dispositivo in oggetto è denominato \textit{portal imager} e consiste in un pannello elettronico montato sul LINAC ed è in grado di generare un'immagine utilizzando raggi di energia dell'ordine dei MeV uscenti dal LINAC allo stesso modo in cui vengono generate le radiografie classiche con i raggi di energia dell'ordine del keV.}. A seguito di queste rilevazioni Yan proponeva di stimare l'errore sistematico paziente-specifico calcolando la media delle deviazioni tra posizione pianificata e posizione di trattamento rilevate nelle prime cinque sedute e di adattare il piano spostando il centro di irradiazione di questa quantità.

\begin{figure}[!t]
\centering
\includegraphics[width=\textwidth]{./cap3/adapt1.png}
\includegraphics[width=\textwidth]{./cap3/adapt2.png}
\caption{Sopra: approccio di adaptive radiotherapy in cui viene unicamente modificato l'isocentro del piano a seguito di rilevazioni dell'errore di posizionamento del paziente tramite portal-imager (PI). Sotto: approccio di adaptive radiotherapy in cui alla modificazione dell'isocentro si aggiunge la modificazione dei margini attorno al CTV in base a rilevazioni di tomografia computerizzata (CT); in questo caso il piano di trattamento si adatta al nuovo PTV \textit{confidence-limit PTV o cl-PTV} stimato dopo le cinque sedute di controllo.}
\label{fig:adaptYAN}
\end{figure}

Pochi anni dopo sempre Yan et al.\cite{Yan2000} pubblicano un altro studio in cui aggiungono alla rilevazione tramite portal-imager anche un rilevazione di tomografia computerizzata del paziente per le prime cinque sedute in modo da osservare il problema del movimento d'organo oltre all'errore di setup. Questo permise agli autori di definire un PTV paziente-specifico che veniva calcolato con la procedura seguente:
\begin{itemize}
\item Si delineava il CTV sui cinque studi CT acquisite in sede delle prime cinque frazioni di trattamento.
\item Si uniscono i contorni dei cinque CTV con il CTV di pianficazione a costituire il cosiddetto \textit{CTV-hull}.
\item Si espande il \textit{CTV-hull} del margine corrispondente all'errore sistematico misurato tramite le cinque rilevazioni tramite portal-imager per formare il \textit{confidence-limit PTV o cl-PTV}. 
\end{itemize}
Entrambi gli approcci adaptive presentati in questa sezione sono raffigurati schematicamente nella Fig.\ref{fig:adaptYAN}.

\`E stato dimostrato che la riduzione del PTV e la conseguente ripianficazione possibile grazie all'applicazione di queste tecniche di adaptive radiotherapy ha portato ad un decremento significativo delle tossicità per gli organi sani, preservando il controllo della malattia \cite{Park2012}. Questo dato è facilmente comprensibile con il paradigma dell'arancia di Verellen \cite{Verellen2007} illustrato in Fig.\ref{fig:verellen}.
\begin{figure}[!t]
\centering
\includegraphics[width=.6\textwidth]{./cap3/Verellen.png}
\caption{Il paradigma dell'arancia di Verellen \cite{Verellen2007} per illustrare l'importanza della riduzione dei margini in radioterapia. In pratica il volume occupato dalla buccia di un arancia è comparabile con il volume dell'arancia stessa. La riduzione anche di pochi millimetri del PTV può comportare una riduzione del volume da irradiare considerevole che si traduce in un maggior risparmio dei tessuti sani.}
\label{fig:verellen}
\end{figure}



\section{Il concetto attuale di \textit{adaptive-radiotherapy} implementato in RayStation}
Il concetto di radioterapia adattiva ha subito negli anni vari aggiornamenti e revisioni ma solo negli ultimi anni è divenuto oggetto di grande interesse. Ciò è dovuto principalmente all'avvento  delle tecniche di imaging per il controllo volumetrico del posizionamento del paziente installate sui moderni LINAC\footnote{La tecnica più diffusa implementata nei moderni LINAC è denominata \textit{cone-beam-computed-tomography o CBCT}. Essa consiste in un sistema capace di ricostruire immagini tomografiche del paziente utilizzando un fascio conico di raggi X e di confrontarle con lo studio CT di pianificazione.}.

\begin{figure}[!t]
\centering
\centerline{
\includegraphics[width=1.2\textwidth]{./cap3/HN_DVH.png}}
\caption{(a) scansione CT del distretto testa-collo di un paziente con delineato in rosso il target e negli altri colori gli organi a rischio. (b) scansione CBCT di controllo a 2.5 settimane con evidente riduzione di volume del target. (c-d) confronto delle distribuzioni di dose di pianificazione (linea continua) e di valutazione (linea tratteggiata). L'istogramma (d) è detto \textit{istogramma dose-volume o DVH} e rappresenta quantitativamente come la distribuzione di dose copre i volumi delineati nello studio CT. \`E da notare come il cambiamento del paziente comporti una variazione della distribuzione di dose agli organi e al target.}
\label{fig:HN_DVH}
\end{figure}

 Nella Fig.\ref{fig:HN_DVH} è illustrato un tipico esempio di irradiazione del distretto testa-collo di largo interesse per la radioterapia adattiva per i grossi cambiamenti che avvengono in itinere nell'anatomia del paziente. A causa dei rapidi gradienti di dose realizzati con le tecniche ad intensità modulata, piccoli cambiamenti di anatomia del paziente possono avere un largo impatto sulla distribuzione di dose finale sia per il target che per gli organi a rischio (Fig.\ref{fig:HN_DVH}d). \\
 
Il TPS RayStation affronta il problema della radioterapia adattiva offrend una soluzione `all-in-one' che si articola in cinque principali step:
\begin{description}
\item[1) Acquisizione degli studi CBCT:] le CBCT acquisite dal sistema di imaging montato sul LINAC vengono spedite al TPS.
\item[2) Deformazione elastica:] viene effettuata una registrazione elastica tra la CT di pianificazione e le CBCT. Questa operazione consiste nel trovare il campo di deformazione geometrico che, se applicato alla CBCT di controllo,   trasforma quest'ultima nella CT di pianificazione. La trasformazione in oggetto è unicamente geometrica: vengono solo variate le coordinate dei pixel ma non il loro valore in scala di grigi.
\item[3) Calcolo della dose in CBCT:] il piano di trattamento viene trasferito alle CBCT e ricalcolato in modo da ottenere la distribuzione di dose relativa alle specifiche frazioni. 
\item[4) Deformazione delle dosi:] sulla base della registrazione deformabile geometrica viene deformata la distribuzione di dose calcolata in CBCT e riportata sulla CT di pianificazione. In questo modo si possono confrontare le distribuzioni di dose pianificata ed effettivamente erogata sulla CT di pianificazione.
\item[5) Adattamento del piano:] sulla base delle differenze tra le distribuzioni di dose pianificata ed erogata si decide o meno di effettuare una ripianificazione del trattamento.
\end{description}

Tutti questi passaggi (a parte il primo) presentano delle criticità che al giorno d'oggi non sono del tutto superate in maniera condivisa. In particolare, le questioni su cui vi è ancora un largo dibattito nella comunità scientifica possono essere riassunte nei punti seguenti:\\

\noindent\textbf{Deformazione elastica:}
\begin{itemize}
\item Quale accuratezza è raggiungibile?
\item Quale è il miglior modo di effettuare le deformazioni?
\end{itemize}
\textbf{Calcolo della dose in CBCT}
\begin{itemize}
\item Quanto è accurato il calcolo della dose in CBCT?
\item Calibrare o non calibrare la CBCT in termini di scala di grigi - densità?
\end{itemize}
\textbf{Deformazione delle dosi:}
\begin{itemize}
\item Quanto è giusto deformare le dosi applicando la trasformazione geometrica?
\end{itemize}
\textbf{Adattamento del piano:}
\begin{itemize}
\item Sulla base di quale metrica si può decidere la ripianificazione?
\item Qual è il momento migliore per la ripianificazione?
\item Che livello di verifica pre-trattamento è richiesto per il piano adattato?
\end{itemize}

La comunità scientifica non è ancora giunta ad una risposta precisa ed esaustiva per le domande sopra elencate per cui il commissioning del TPS dal punto di vista della radioterapia adattiva risulta tutt'ora (Febbraio 2016) in fase di ottimizzazione. In particolare, gli sforzi che si stanno compiendo in questo senso consistono nella partecipazione a due studi multicentrici nazionali orientati a chiarire gli aspetti più dibattuti ossia quelli riguardanti la deformazione elastica e l'adattamento del piano.

\subsection{Studio multicentrico per l'ottimizzazione del protocollo di deformazione elastica}
Il TPS RayStation implementa un algoritmo di deformazione elastica ibrido \cite{RaySearchLaboratories2014}. In un primo step il campo di deformazione viene calcolato considerando unicamente la differenza tra i livelli di grigio dello studio di riferimento e quello da deformare. In seconda battuta è possibile migliorare la deformazione `guindandola' con dei contorni geometrici (ROI di controllo).

Ciò si realizza matematicamente minimizzando un funzionale cha ha la forma seguente:
\begin{equation}
\label{eq:def}
\begin{split}
F =& \alpha \cdot (1- Correlation\, Coefficient) + \\
   & + \beta \cdot Regularization\,Term + \\
   & + \gamma \cdot ROI\,Penalty\,Term
\end{split}
\end{equation}
dove il \textit{Correlation coefficient} è un termine del tipo coefficiente di correlazione di Pearson e misura la correlazione tra i livelli di grigio delle due immagini (è uguale a 1 per immagini identiche). Il termine di regolarizzazione (\textit{Regularization Term}) aumenta all'aumentare dell'intensità della deformazione e serve a penalizzare soluzioni improbabili dal punto di vista fisico. Infine il \textit{ROI Penalty Term} è un termine che diminuisce quanto più le ROI di controllo tra le due immagini sono sovrapposte.

Nella corrente implementazione dell'algoritmo di deformazione in RayStation (v.4.7) i termini peso $\alpha$, $\beta$ e $\gamma$, il coefficiente di correlazione ed il termine di regolarizzazione sono tutti fissati dalla ditta. Quello che viene lasciato all'operatore è la scelta delle ROI di controllo ossia la forma dell'ultimo termine della Eq.\eqref{eq:def}.

Il razionale dello studio multicentrico cui si è presi parte consiste nell'investigare l'accuratezza di vari algoritmi di registrazione deformabile e di sviluppare le metodologie che portano alla migliore deformazione possibile. A questo proposito è stato riunito un gruppo di 13 istituzioni italiane (Gruppo YES) per un totale di 5 diverse soluzioni commerciali per la registrazione elastica, tra cui il TPS RayStation. Lo studio si articola in più fasi ed è tutt'ora (Febbraio 2016) in corso di svolgimento. Nell'ambito di questo lavoro di tesi discuteremo i risultati raggiunti nella fase riguardante i casi clinici.

Per la fase clinica dello studio YES sono stati reclutati 3 studi CT per i distretti pelvico, testa-collo e polmonare (Fig.\ref{fig:YES_sites}).
\begin{figure}
\centering
\includegraphics[width=\textwidth]{./cap3/YES_Sites.png}
\caption{Siti anatomici studiati nell'ambito del gruppo YES}
\label{fig:YES_sites}
\end{figure}
Per ogni distretto è stato creato un altro studio CT in cui è stata simulate tramite software (ImSimQA) una deformazione che tipicamente si realizza nella routine clinica:
\begin{description}
\item[Distretto pelvico:] riempimento vescicale minore rispetto alla CT di pianificazione.
\item[Distretto testa-collo:] riduzione di volume delle parotidi.
\item[Distretto polmonare:] riduzione di volume del tumore.
\end{description}
Ogni istituzione aveva il compito di ricavare in maniera inversa la deformazione simulata utilizzando gli strumenti in proprio possesso. Sulla base della deformazione ottenuta veniva richiesto di mappare delle regioni di interesse (ROI) dalla CT iniziale alla CT deformata. Le ROI mappate da ogni centro sono state poi confrontate con le ROI `gold' deformate secondo la deformazione simulata, nota solo al centro coordinatore.\\
L'indicatore utilizzato per confrontare le ROI mappate dai centri con le ROI `gold' è l'indice di conformità di Jaccard:
%\begin{equation}
%J(ROI_{A},ROI_{B{) = \frac{|ROI_{A} \cap ROI_{B}|}{|ROI_{A} \cup ROI_{B}|}
%\end{equation}
Questo indice è uguale a zero per ROI disgiunte mentre è uguale a uno per ROI esattamente sovrapposte.
I risultati raggiunti a livello multicentrico sono riassunti nella Fig.\ref{fig:YES_results} e possono essere riassunti principalmente nei seguenti punti:
\begin{itemize}
\item La differenza tra le performance globali dei vari centri non è risultata statisticamente significativa tranne che per un centro (Fig.\ref{fig:YES_results}a). 
\item La deformazione è risultata statisticamente più accurata  nel caso torace piuttosto che i casi pelvici e testa-collo (Fig.\ref{fig:YES_results}b).
\end{itemize}
\begin{figure}
\centering
\includegraphics[width=.49\textwidth]{./cap3/YES_results_a.png}
\includegraphics[width=.49\textwidth]{./cap3/YES_results_b.png}
\caption{(a) risultati del confronto per quanto riguarda l'accuratezza globale delle deformazioni effettuate dai singoli centri. (b) risultati dell'accuratezza raggiunta considerando tutti i centri a seconda del sito di interesse. I valori statisticamente differenti dal campione sono cerchiati in rosso. Il test statistico utilizzato è del tipo ANOVA (\textit{ANalysis Of VAriances}).}
\label{fig:YES_results}
\end{figure}
Dal primo risultato si può evincere che le soluzioni commerciali esistenti portano a soluzioni di precisione comparabile pur implementando algoritmi differenti tra loro. Le performance centro statisticamente differente sono imputabili ad differente livello di training degli operatori coinvolti.

Il secondo risultato è spiegabile con la peculiarità del sito toracico (tumore polmonare) rispetto ai siti testa-collo e pelvico. In questi ultimi, molti dei tessuti coinvolti risultano essere differenti di pochi livelli nella scala di grigi (basso contrasto) per cui è necessario guidare la deformazione con varie ROI di controllo (il coefficiente di correlazione dei grigi non è sufficiente ad ottenere una deformazione accurata). Diversa è la situazione per il caso toracico in cui il tessuto tumorale all'interno del polmone risulta essere ad alto contrasto con il tessuto circostante per cui la deformazione risulta semplice da trovare da parte dell'algoritmo, senza l'ausilio di particolari `aiuti'.

Continuare qui parlando dei risultati ottenuti a livello locale e della bozza dei protocolli di deformazione


\subsection{Studio multicentrico per predire la necessità di adattare il piano}
Continua qui prendi i risultati dell'AIFM etc etc
sottosezione adaptive (GUIDI) con risultati preliminari tipo quella roba delle parotidi etc etc e siamo quasi quasi per finire :OOO










%\chapter{Misure NMR}
\minitoc
\textsf{In questo capitolo vengono presentate....}\\



\section{Conclusioni}
In questo capitolo abbiamo mostrato...























%\chapter{Conclusioni e Sviluppi Futuri}
\minitoc
\textsf{In questo capitolo conclusivo della tesi vengono riassunti tutti i }



\chapter*{Conclusioni}
\addstarredchapter{Conclusioni}
Grazie al lavoro di questa tesi abbiamo dimostrato ....

I vari step e gli obiettivi raggiunti nel corso della tesi che hanno permesso di ottenere i suddetti risultati, sono riassunti sommariamente nell'elenco seguente:
\begin{itemize}
\item \`E stata studiata 
\end{itemize}



\chapter*{Ringraziamenti}
\pagestyle{plain}
Desidero ringraziare sentitamente .\\

\noindent Intendo poi ringraziare il\\

\noindent Ringrazio anche \\

































%\appendix
%

\chapter{Programmi RADIA}
%%%%%%INTESTAZIONE ELEGANTE%%%%%%
\pagestyle{fancy}
% with this we ensure that the chapter and section
% headings are in lowercase.
\renewcommand{\chaptermark}[1]{%
\markboth{#1}{}}
\renewcommand{\sectionmark}[1]{%
\markright{\thesection\ #1}}
\fancyhf{} % delete current header and footer
\fancyhead[LE,RO]{\bfseries\thepage}
\fancyhead[LO]{\bfseries\rightmark}
\fancyhead[RE]{\bfseries\leftmark}
\renewcommand{\headrulewidth}{0.5pt}
\renewcommand{\footrulewidth}{0pt}
\addtolength{\headheight}{2.5pt} % space for the rule
\fancypagestyle{plain}{%
\fancyhead{} % get rid of headers on plain pages
\renewcommand{\headrulewidth}{0pt} }
%%%%%%INTESTAZIONE ELEGANTE%%%%%%

Ciao

%%%%%%BIBLIOGRAFIA%%%%%%
\printbibliography[heading=bibintoc]
%%\begin{thebibliography}{100}
%	\bibitem{haa} E.Mark Haacke: \emph{Magnetic Resonance Imaging - Physical Principles and Sequence Design} (Wiley - Liss, 1999).
%
%\end{thebibliography}

\newpage
\printbibliography[heading=bibintoc]
\end{document}
