
\chapter*{Ringraziamenti}
Questo lavoro di tesi rappresenta la conclusione di un percorso durato più di quattro anni che ha rappresentato un passo fondamentale ed indelebile della mia vita. Tutto ciò è stato possibile grazie alle persone che mi hanno accompagnato, istruito ed assistito in questo cammino. Le parole che vi rivolgo non rendono onore all'importanza che avete rivestito per me.

\vspace*{2ex}

Desidero ringraziare il dr.Giovanni Orlandi, direttore della U.O.C. di Fisica Sanitaria della ASL di Teramo sul piano umano e professionale, per avermi accolto nella sua unità, per avermi subito fatto sentire parte del gruppo, per tutto il tempo dedicato a guidarmi, istruirmi ed educarmi ad esercitare questa professione, per essere stato relatore di questa tesi, per tutti i pomeriggi spesi nelle correzioni e nei consigli e per l'umanità che mi ha sempre dimostrato in ogni occasione.

\vspace*{2ex}

Desidero inoltre ringraziare i dirigenti fisici (presenti ed ex) della U.O.C. di Fisica Sanitaria che hanno rappresentato la mia seconda famiglia: dr.Nico Adorante, dr.ssa Floriana Bartolucci, dr.Christian Fidanza e dr.ssa Federica Rosica, per tutta la loro considerazione nei miei confronti, per aver contribuito in maniera determinante alla mia formazione, per tutto il tempo speso a dispensare consigli, a correggere i miei errori, a trasferire il loro sapere e la loro esperienza alla mia persona. Li ringrazio inoltre per la grande umanità dimostratami in ogni momento durante tutti questi anni.

\vspace*{2ex}

Ringrazio poi i tecnici della U.O.C. di Fisica Sanitaria dr. Daniele Ciuffetelli e dr.ssa Antonella Rastelli per la loro immensa fraternità nei miei confronti, per tutti i consigli e per aver condiviso con me momenti indimenticabili sia lavorativi che extra-lavorativi.

\vspace*{2ex}

Ringrazio la dr.ssa Silvia Colacicchi direttore della scuola di specializzazione in Fisica Medica dell'Università dell'Aquila, sempre pronta ad ascoltare ogni mia esigenza e disponibile verso ogni mia richiesta. Con essa ringrazio anche tutti i professori del corso di specializzazione, per avermi formato e fornito gli strumenti teorici necessari all'esercizio di questa professione.

\vspace*{2ex}

Ringrazio poi il direttore della U.O.C. di Radioterapia dr. Carlo D'Ugo per avermi dato accesso pieno ed incondizionato alle strutture e apparecchiature della sua unità operativa e per aver permesso la conduzione di vari studi determinanti per il mio sapere e per il mio curriculum professionale.

\vspace*{2ex}

Ringrazio il personale medico della U.O.C. di Radioterapia dr. Sergio Buonopane, dr. Marco De Nicolò, dr.ssa Milena di Genesio Pagliuca, dr. Saverio Malara e dr.ssa Caterina Montoro per la loro considerazione nei miei confronti e per le proficue discussioni sia lavorative che extra-lavorative. Un particolare ringraziamento va a Saverio per la considerazione fraterna che mi ha sempre riservato e Milena per la sua presenza e simpatia insostituibile e per aver costantemente e quotidianamente accresciuto la mia autostima :-).

\vspace*{2ex}

Ringrazio il personale tecnico e infermieristico della U.O.C. di Radioterapia per avermi accolto come in famiglia, per tutti gli inviti a cena e per ogni momento conviviale passato assieme in spensieratezza.

\vspace*{2ex}

Ringrazio i miei compagni di specializzazione Emidio Bellabarba, Stefania Fabiani, Serena Fattori, Stefano Giancaterino, Martina Paoli e Mattia Polsoni, per i viaggi assieme, per i passaggi e per tutto il tempo passato assieme sia a studiare che a divertirci.

\vspace*{2ex}

Ringrazio infine la mia famiglia per la quale non spenderò molte parole. Grazie per aver permesso ai miei sogni di divenire realtà.

\vspace*{2ex}
\begin{flushright}
Teramo, 01/07/2016
\end{flushright}







 